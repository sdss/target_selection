\begin{center}\rule{0.5\linewidth}{0.5pt}\end{center}

\hypertarget{bhm_aqmes_med_plan0.5.0}{%
\subsection{bhm\_aqmes\_med}\label{bhm_aqmes_med_plan0.5.0}}

\noindent\textbf{target\_selection plan:} 0.5.0

\noindent\textbf{target\_selection tag:}
\href{https://github.com/sdss/target_selection/tree/0.3.0/}{0.3.0}

\noindent\textbf{Summary:} Spectroscopically confirmed optically bright SDSS
QSOs, selected from the SDSS QSO catalogue (DR16Q,
\citealt{Lyke2020}). Located in 36 mostly disjoint fields within the SDSS QSO
footprint that were pre-selected to contain higher than average numbers
of bright QSOs and CSC targets. The list of field centres can be found
within
\href{https://github.com/sdss/target_selection/blob/0.3.0/python/target_selection/masks/candidate_target_fields_bhm_aqmes_med_v0.3.1.fits}{the
target\_selection repository}.

\noindent\textbf{Simplified description of selection criteria:} Select all
objects from SDSS DR16 QSO catalogue that have
16.0\textless{}sdss\_psfmag\_i\textless{}19.1~AB, that lie within
1.49~degrees of at least one AQMES-medium field location.


\noindent\textbf{Target priority options:} 1100

\noindent\textbf{Cadence options:} dark\_10x4\_4yr

\noindent\textbf{Implementation:}
\href{https://github.com/sdss/target_selection/blob/0.3.0/python/target_selection/cartons/bhm_aqmes.py}{bhm\_aqmes.py}

\noindent\textbf{Number of targets:} 2663

\begin{center}\rule{0.5\linewidth}{0.5pt}\end{center}

\hypertarget{bhm_aqmes_med_faint_plan0.5.0}{%
\subsection{bhm\_aqmes\_med\_faint}\label{bhm_aqmes_med_faint_plan0.5.0}}

\noindent\textbf{target\_selection plan:} 0.5.0

\noindent\textbf{target\_selection tag:}
\href{https://github.com/sdss/target_selection/tree/0.3.0/}{0.3.0}

\noindent\textbf{Summary:} Spectroscopically confirmed optically faint SDSS QSOs,
selected from the SDSS QSO catalogue (DR16Q,
\citealt{Lyke2020}). Located in 36 mostly disjoint fields within the SDSS QSO
footprint that were pre-selected to contain higher than average numbers
of bright QSOs and CSC targets. The list of field centres can be found
within
\href{https://github.com/sdss/target_selection/blob/0.3.0/python/target_selection/masks/candidate_target_fields_bhm_aqmes_med_v0.3.1.fits}{the
target\_selection repository}.

\noindent\textbf{Simplified description of selection criteria:} Select all
objects from SDSS DR16 QSO catalogue that have
19.1\textless{}sdss\_psfmag\_i\textless{}21.0~AB, that lie within
1.49~degrees of at least one AQMES-medium field location.


\noindent\textbf{Target priority options:} 3100

\noindent\textbf{Cadence options:} dark\_10x4\_4yr

\noindent\textbf{Implementation:}
\href{https://github.com/sdss/target_selection/blob/0.3.0/python/target_selection/cartons/bhm_aqmes.py}{bhm\_aqmes.py}

\noindent\textbf{Number of targets:} 16853

\begin{center}\rule{0.5\linewidth}{0.5pt}\end{center}

\hypertarget{bhm_aqmes_wide2_plan0.5.4}{%
\subsection{bhm\_aqmes\_wide2}\label{bhm_aqmes_wide2_plan0.5.4}}

\noindent\textbf{target\_selection plan:} 0.5.4

\noindent\textbf{target\_selection tag:}
\href{https://github.com/sdss/target_selection/tree/0.3.5/}{0.3.5}

\noindent\textbf{Summary:} Spectroscopically confirmed optically bright SDSS
QSOs, selected from the SDSS QSO catalogue (DR16Q,
\citealt{Lyke2020}). Located in 425 fields within the SDSS QSO footprint,
where the choice of survey area prioritized field that overlapped with
the SPIDERS footprint (approx 180\textless{}b\textless{}360~deg), and/or
had higher than average numbers of bright QSOs and CSC targets. The list
of field centres can be found
\href{https://github.com/sdss/target_selection/blob/0.3.0/python/target_selection/masks/candidate_target_fields_bhm_aqmes_wide_v0.3.1.fits}{within
the target\_selection repository}.

\noindent\textbf{Simplified description of selection criteria:} Select all
objects from SDSS DR16 QSO catalogue that have
16.0\textless{}sdss\_psfmag\_i\textless{}19.1~AB, and that lie within
1.49~degrees of at least one AQMES-wide field location.


\noindent\textbf{Target priority options:} 1210, 1211

\noindent\textbf{Cadence options:} dark\_2x4

\noindent\textbf{Implementation:}
\href{https://github.com/sdss/target_selection/blob/0.3.5/python/target_selection/cartons/bhm_aqmes.py}{bhm\_aqmes.py}

\noindent\textbf{Number of targets:} 24142

\begin{center}\rule{0.5\linewidth}{0.5pt}\end{center}

\hypertarget{bhm_aqmes_wide2_faint_plan0.5.4}{%
\subsection{bhm\_aqmes\_wide2\_faint}\label{bhm_aqmes_wide2_faint_plan0.5.4}}

\noindent\textbf{target\_selection plan:} 0.5.4

\noindent\textbf{target\_selection tag:}
\href{https://github.com/sdss/target_selection/tree/0.3.5/}{0.3.5}

\noindent\textbf{Summary:} Spectroscopically confirmed optically faint SDSS QSOs,
selected from the SDSS QSO catalogue (DR16Q,
\citealt{Lyke2020}). Located in 425 fields within the SDSS QSO footprint,
where the choice of survey area prioritized field that overlapped with
the SPIDERS footprint (approx 180\textless{}b\textless{}360~deg), and/or
had higher than average numbers of bright QSOs and CSC targets. The list
of field centres can be found
\href{https://github.com/sdss/target_selection/blob/0.3.0/python/target_selection/masks/candidate_target_fields_bhm_aqmes_wide_v0.3.1.fits}{within
the target\_selection repository}.

\noindent\textbf{Simplified description of selection criteria:} Select all
objects from SDSS DR16 QSO catalogue that have
19.1\textless{}sdss\_psfmag\_i\textless{}21.0~AB, and that lie within
1.49~degrees of at least one AQMES-wide field location.


\noindent\textbf{Target priority options:} 3210, 3211

\noindent\textbf{Cadence options:} dark\_2x4

\noindent\textbf{Implementation:}
\href{https://github.com/sdss/target_selection/blob/0.3.5/python/target_selection/cartons/bhm_aqmes.py}{bhm\_aqmes.py}

\noindent\textbf{Number of targets:} 99586

\begin{center}\rule{0.5\linewidth}{0.5pt}\end{center}

\hypertarget{bhm_aqmes_bonus_core_plan0.5.4}{%
\subsection{bhm\_aqmes\_bonus\_core}\label{bhm_aqmes_bonus_core_plan0.5.4}}

\noindent\textbf{target\_selection plan:} 0.5.4

\noindent\textbf{target\_selection tag:}
\href{https://github.com/sdss/target_selection/tree/0.3.5/}{0.3.5}

\noindent\textbf{Summary:} Spectroscopically confirmed optically bright SDSS
QSOs, selected from the SDSS QSO catalogue (DR16Q,
\citealt{Lyke2020}). Located anywhere within the SDSS DR16Q footprint.

\noindent\textbf{Simplified description of selection criteria:} Select all
objects from SDSS DR16 QSO catalogue that have
16.0\textless{}sdss\_psfmag\_i\textless{}19.1~AB


\noindent\textbf{Target priority options:} 3300,3301

\noindent\textbf{Cadence options:} dark\_1x4

\noindent\textbf{Implementation:}
\href{https://github.com/sdss/target_selection/blob/0.3.5/python/target_selection/cartons/bhm_aqmes.py}{bhm\_aqmes.py}

\noindent\textbf{Number of targets:} 83163

\begin{center}\rule{0.5\linewidth}{0.5pt}\end{center}

\hypertarget{bhm_aqmes_bonus_faint_plan0.5.4}{%
\subsection{bhm\_aqmes\_bonus\_faint}\label{bhm_aqmes_bonus_faint_plan0.5.4}}

\noindent\textbf{target\_selection plan:} 0.5.4

\noindent\textbf{target\_selection tag:}
\href{https://github.com/sdss/target_selection/tree/0.3.5/}{0.3.5}

\noindent\textbf{Summary:} Spectroscopically confirmed optically faint SDSS QSOs,
selected from the SDSS QSO catalogue (DR16Q,
\citealt{Lyke2020}). Located anywhere within the SDSS DR16Q footprint.

\noindent\textbf{Simplified description of selection criteria:} Select all
objects from SDSS DR16 QSO catalogue that have
19.1\textless{}sdss\_psfmag\_i\textless{}21.0~AB


\noindent\textbf{Target priority options:} 3302,3303

\noindent\textbf{Cadence options:} dark\_1x4

\noindent\textbf{Implementation:}
\href{https://github.com/sdss/target_selection/blob/0.3.5/python/target_selection/cartons/bhm_aqmes.py}{bhm\_aqmes.py}

\noindent\textbf{Number of targets:} 424163

\begin{center}\rule{0.5\linewidth}{0.5pt}\end{center}

\hypertarget{bhm_aqmes_bonus_bright_plan0.5.4}{%
\subsection{bhm\_aqmes\_bonus\_bright}\label{bhm_aqmes_bonus_bright_plan0.5.4}}

\noindent\textbf{target\_selection plan:} 0.5.4

\noindent\textbf{target\_selection tag:}
\href{https://github.com/sdss/target_selection/tree/0.3.5/}{0.3.5}

\noindent\textbf{Summary:} Spectroscopically confirmed, extremely optically
bright SDSS QSOs, selected from the SDSS QSO catalogue (DR16Q,
\citealt{Lyke2020}). Located anywhere within the SDSS DR16Q footprint.

\noindent\textbf{Simplified description of selection criteria:} Select all
objects from SDSS DR16 QSO catalogue that have
14.0\textless{}sdss\_psfmag\_i\textless{}18.0~AB


\noindent\textbf{Target priority options:} 4040,4041

\noindent\textbf{Cadence options:} bright\_3x1

\noindent\textbf{Implementation:}
\href{https://github.com/sdss/target_selection/blob/0.3.5/python/target_selection/cartons/bhm_aqmes.py}{bhm\_aqmes.py}

\noindent\textbf{Number of targets:} 10848

\begin{center}\rule{0.5\linewidth}{0.5pt}\end{center}

\hypertarget{bhm_rm_ancillary_plan0.5.0}{%
\subsection{bhm\_rm\_ancillary}\label{bhm_rm_ancillary_plan0.5.0}}

\noindent\textbf{target\_selection plan:} 0.5.0

\noindent\textbf{target\_selection tag:}
\href{https://github.com/sdss/target_selection/tree/0.3.0/}{0.3.0}

\noindent\textbf{Summary:} A supporting sample of candidate QSOs which have been
selected by the Gaia-unWISE AGN catalog
(\citealt{Shu2019}) and/or the SDSS XDQSO catalog
(\citealt{Bovy2011}). These targets are located within five (+1 backup) well
known survey fields (SDSS-RM, COSMOS, XMM-LSS, ECDFS, CVZ-S/SEP, and
ELIAS-S1).

\noindent\textbf{Simplified description of selection criteria:} Starting from a
parent catalogue of optically selected objects in the RM fields (as
presented by
\citealt{Yang2022}), select candidate QSOs that satisfy all of the
following: i) are identified via external ancillary methods
(photo\_bitmask \& 3 != 0); ii) have
15\textless{}psfmag\_i\textless{}21.5~AB
(16\textless{}G\textless{}21.7~AB in the CVZ-S/SEP field); iii) do not
have significant detections (\textgreater{}3$\sigma$) of parallax and/or proper
motion in Gaia DR2; iv) are not vetoed due to results of visual
inspections of recent spectroscopy; and v) do not lie in the SDSS-RM
field


\noindent\textbf{Target priority options:} 900-1050

\noindent\textbf{Cadence options:} dark\_174x8, dark\_100x8

\noindent\textbf{Implementation:}
\href{https://github.com/sdss/target_selection/blob/0.3.0/python/target_selection/cartons/bhm_rm.py}{bhm\_rm.py}

\noindent\textbf{Number of targets:} 943

\begin{center}\rule{0.5\linewidth}{0.5pt}\end{center}

\hypertarget{bhm_rm_core_plan0.5.0}{%
\subsection{bhm\_rm\_core}\label{bhm_rm_core_plan0.5.0}}

\noindent\textbf{target\_selection plan:} 0.5.0

\noindent\textbf{target\_selection tag:}
\href{https://github.com/sdss/target_selection/tree/0.3.0/}{0.3.0}

\noindent\textbf{Summary:} A sample of candidate QSOs selected via the methods
presented by
\citet{Yang2022}. These targets are located within five (+1 backup) well
known survey fields (SDSS-RM, COSMOS, XMM-LSS, ECDFS, CVZ-S/SEP, and
ELIAS-S1).

\noindent\textbf{Simplified description of selection criteria:} Starting from a
parent catalogue of optically selected objects in the RM fields (as
presented by
\citealt{Yang2022}), select candidate QSOs that satisfy all of the
following: i) are identified via the Skew-T algorithm (skewt\_qso == 1);
ii) have 17\textless{}psfmag\_i\textless{}21.5~AB
(16\textless{}G\textless{}21.7~AB in the CVZ-S/SEP field); iii) do not
have significant detections (\textgreater{}3$\sigma$) of parallax and/or proper
motion in Gaia DR2; iv) are not vetoed due to results of visual
inspections of recent spectroscopy; vi) have detections in all of the
gri bands (a Gaia detection is sufficient in the CVZ-S/SEP field);and
vii) do not lie in the SDSS-RM field


\noindent\textbf{Target priority options:} 900-1050

\noindent\textbf{Cadence options:} dark\_174x8, dark\_100x8

\noindent\textbf{Implementation:}
\href{https://github.com/sdss/target_selection/blob/0.3.0/python/target_selection/cartons/bhm_rm.py}{bhm\_rm.py}

\noindent\textbf{Number of targets:} 3721

\begin{center}\rule{0.5\linewidth}{0.5pt}\end{center}

\hypertarget{bhm_rm_var_plan0.5.0}{%
\subsection{bhm\_rm\_var}\label{bhm_rm_var_plan0.5.0}}

\noindent\textbf{target\_selection plan:} 0.5.0

\noindent\textbf{target\_selection tag:}
\href{https://github.com/sdss/target_selection/tree/0.3.0/}{0.3.0}

\noindent\textbf{Summary:} A sample of candidate QSOs selected via their optical
variability properties, as presented by
\citet{Yang2022}. These targets are located within five (+1 backup) well
known survey fields (SDSS-RM, COSMOS, XMM-LSS, ECDFS, CVZ-S/SEP, and
ELIAS-S1).

\noindent\textbf{Simplified description of selection criteria:} Starting from a
parent catalogue of optically selected objects in the RM fields (as
presented by
\citealt{Yang2022}), select candidate QSOs that satisfy all of the
following: i) have significant variability in the DES or PanSTARRS1
multi-epoch photometry (var\_sn{[}g{]}\textgreater{}3 and
var\_rms{[}g{]}\textgreater{}0.05); ii) have
17\textless{}psfmag\_i\textless{}20.5~AB
(16\textless{}G\textless{}21.7~AB in the CVZ-S/SEP field); iii) do not
have significant detections (\textgreater{}3$\sigma$) of parallax and/or proper
motion in Gaia DR2; iv) are not vetoed due to results of visual
inspections of recent spectroscopy; and vi) do not lie in the SDSS-RM
field


\noindent\textbf{Target priority options:} 900-1050

\noindent\textbf{Cadence options:} dark\_174x8, dark\_100x8

\noindent\textbf{Implementation:}
\href{https://github.com/sdss/target_selection/blob/0.3.0/python/target_selection/cartons/bhm_rm.py}{bhm\_rm.py}

\noindent\textbf{Number of targets:} 934

\begin{center}\rule{0.5\linewidth}{0.5pt}\end{center}

\hypertarget{bhm_rm_known_spec_plan0.5.0}{%
\subsection{bhm\_rm\_known\_spec}\label{bhm_rm_known_spec_plan0.5.0}}

\noindent\textbf{target\_selection plan:} 0.5.0

\noindent\textbf{target\_selection tag:}
\href{https://github.com/sdss/target_selection/tree/0.3.0/}{0.3.0}

\noindent\textbf{Summary:} A sample of known QSOs identified through optical
spectroscopy from various projects, as collated by
\citet{Yang2022}. These targets are located within five (+1 backup) well
known survey fields (SDSS-RM, COSMOS, XMM-LSS, ECDFS, CVZ-S/SEP, and
ELIAS-S1).

\noindent\textbf{Simplified description of selection criteria:} Starting from a
parent catalogue of optically selected objects in the RM fields (as
presented by
\citealt{Yang2022}), select targets which satisfy all of the following: i)
are flagged as having a spectroscopic identification (in the parent
catalogue or in the bhm\_rm\_tweaks table); ii) have
15\textless{}psfmag\_i\textless{}21.7~AB (SDSS-RM, CDFS, ELIAS-S1
field), 16\textless{}G\textless{}21.7~Vega (CVZ-S/SEP field),
15\textless{}psfmag\_i\textless{}21.5~AB (COSMOS and XMM-LSS fields);
iii) have a spectroscopic redshift in the range
0.005\textless{}z\textless{}7; iv) are not vetoed due to results of
visual inspections of recent spectroscopy


\noindent\textbf{Target priority options:} 900-1050

\noindent\textbf{Cadence options:} dark\_174x8, dark\_100x8

\noindent\textbf{Implementation:}
\href{https://github.com/sdss/target_selection/blob/0.3.0/python/target_selection/cartons/bhm_rm.py}{bhm\_rm.py}

\noindent\textbf{Number of targets:} 3022

\begin{center}\rule{0.5\linewidth}{0.5pt}\end{center}

\hypertarget{bhm_csc_apogee_plan0.5.15}{%
\subsection{bhm\_csc\_apogee}\label{bhm_csc_apogee_plan0.5.15}}

\noindent\textbf{target\_selection plan:} 0.5.15

\noindent\textbf{target\_selection tag:}
\href{https://github.com/sdss/target_selection/tree/0.3.14/}{0.3.14}

\noindent\textbf{Summary:} X-ray sources from the CSC2.0 source catalogue with
NIR counterparts in 2MASS PSC

\noindent\textbf{Simplified description of selection criteria:} Starting from the
parent catalogue of CSC sources with optical/IR counterparts
(bhm\_csc\_v2). Select entries satisfying the following criteria: i) NIR
counterpart is from the 2MASS catalogue, ii) 2MASS H-band magnitude
measurement is not null and in the accepted range for SDSS-V:
7.0\textless{}H\textless{}14.0. Allocate cadence (exposure time)
requests based on H-band magnitude.


\noindent\textbf{Target priority options:} 2930-2939

\noindent\textbf{Cadence options:} bright\_1x1,bright\_3x1

\noindent\textbf{Implementation:}
\href{https://github.com/sdss/target_selection/blob/0.3.14/python/target_selection/cartons/bhm_csc.py}{bhm\_csc.py}

\noindent\textbf{Number of targets:} 48928

\begin{center}\rule{0.5\linewidth}{0.5pt}\end{center}

\hypertarget{bhm_csc_boss_plan0.5.15}{%
\subsection{bhm\_csc\_boss}\label{bhm_csc_boss_plan0.5.15}}

\noindent\textbf{target\_selection plan:} 0.5.15

\noindent\textbf{target\_selection tag:}
\href{https://github.com/sdss/target_selection/tree/0.3.14/}{0.3.14}

\noindent\textbf{Summary:} X-ray sources from the CSC2.0 source catalogue with
counterparts in Panstarrs1-DR1 or Gaia DR2.

\noindent\textbf{Simplified description of selection criteria:} Starting from the
parent catalogue of CSC sources with optical/IR counterparts
(bhm\_csc\_v2). Select entries satisfying the following criteria: i)
optical counterpart is from the PanSTARRS1 or Gaia dr2 catalogues, ii)
optical flux/magnitude is in the accepted range for SDSS-V:
psfmag\_{[}g,r,i,z{]}\textgreater{}13.5~AB and non-Null psfmag\_i
(objects with PanSTARRS1 counterparts); G,RP\textgreater{}13.0~Vega
(Gaia DR2 counterparts). Deprioritize targets which already have good
quality SDSS spectroscopy. Allocate cadence (exposure time) requests
based on optical brightness (PS1 psfmag\_i or Gaia G).


\noindent\textbf{Target priority options:} 1920-1939, 2920-2939

\noindent\textbf{Cadence options:} bright\_1x1,dark\_1x2,dark\_1x4

\noindent\textbf{Implementation:}
\href{https://github.com/sdss/target_selection/blob/0.3.14/python/target_selection/cartons/bhm_csc.py}{bhm\_csc.py}

\noindent\textbf{Number of targets:} 122731

\begin{center}\rule{0.5\linewidth}{0.5pt}\end{center}

\hypertarget{bhm_gua_bright_plan0.5.0}{%
\subsection{bhm\_gua\_bright}\label{bhm_gua_bright_plan0.5.0}}

\noindent\textbf{target\_selection plan:} 0.5.0

\noindent\textbf{target\_selection tag:}
\href{https://github.com/sdss/target_selection/tree/0.3.0/}{0.3.0}

\noindent\textbf{Summary:} A sample of optically bright candidate AGN lacking
spectroscopic confirmations, derived from the parent sample presented by
\citet{Shu2019}, who applied a machine-learning approach to select QSO
candidates from a combination of the Gaia DR2 and unWISE catalogues.

\noindent\textbf{Simplified description of selection criteria:} Starting with the
\citet{Shu2019} catalogue, select targets which satisfy the following
criteria: i) have a Random Forest probability of being a QSO
of\textgreater{}0.8, ii) are in the magnitude range suitable for BOSS
spectroscopy in bright time (G\textgreater{}13.0 and
RP\textgreater{}13.5, as well as G\textless{}18.5 or
RP\textless{}18.5,~Vega mags), iii) do not have good optical
spectroscopic measurements from a previous iteration of SDSS.


\noindent\textbf{Target priority options:} 4040

\noindent\textbf{Cadence options:} bright\_2x1

\noindent\textbf{Implementation:}
\href{https://github.com/sdss/target_selection/blob/0.3.0/python/target_selection/cartons/bhm_gua.py}{bhm\_gua.py}

\noindent\textbf{Number of targets:} 254601

\begin{center}\rule{0.5\linewidth}{0.5pt}\end{center}

\hypertarget{bhm_gua_dark_plan0.5.0}{%
\subsection{bhm\_gua\_dark}\label{bhm_gua_dark_plan0.5.0}}

\noindent\textbf{target\_selection plan:} 0.5.0

\noindent\textbf{target\_selection tag:}
\href{https://github.com/sdss/target_selection/tree/0.3.0/}{0.3.0}

\noindent\textbf{Summary:} A sample of optically faint candidate AGN lacking
spectroscopic confirmations, derived from the parent sample presented by
\citet{Shu2019}, who applied a machine-learning approach to select QSO
candidates from a combination of the Gaia DR2 and unWISE catalogues.

\noindent\textbf{Simplified description of selection criteria:} Starting with the
\citet{Shu2019} catalogue, select targets which satisfy the following
criteria: i) have a Random Forest probability of being a QSO
of\textgreater{}0.8, ii) are in the magnitude range suitable for BOSS
spectroscopy in dark time (G\textgreater{}16.5 and RP\textgreater{}16.5,
as well as G\textless{}21.2 or RP\textless{}21.0,~Vega mag), iii) do not
have good optical spectroscopic measurements from a previous iteration
of SDSS.


\noindent\textbf{Target priority options:} 3400

\noindent\textbf{Cadence options:} dark\_1x4

\noindent\textbf{Implementation:}
\href{https://github.com/sdss/target_selection/blob/0.3.0/python/target_selection/cartons/bhm_gua.py}{bhm\_gua.py}

\noindent\textbf{Number of targets:} 2156582

\begin{center}\rule{0.5\linewidth}{0.5pt}\end{center}

\hypertarget{bhm_colr_galaxies_lsdr8_plan0.5.16}{%
\subsection{bhm\_colr\_galaxies\_lsdr8}\label{bhm_colr_galaxies_lsdr8_plan0.5.16}}

\noindent\textbf{target\_selection plan:} 0.5.16

\noindent\textbf{target\_selection tag:}
\href{https://github.com/sdss/target_selection/tree/0.3.13/}{0.3.13}

\noindent\textbf{Summary:} A supplementary magnitude limited sample of optically
bright galaxies selected from the legacysurvey.org/dr8 optical/IR
imaging catalogue. Selection is based on optical morphology, lack of
Gaia DR2 parallax, and several magnitude cuts.

\noindent\textbf{Simplified description of selection criteria:} Starting from the
legacy\_survey\_dr8 catalogue (lsdr8), select entries satisfying all of
the following criteria: i) lsdr8 morphological type != 'PSF', ii) zero
or Null parallax in Gaia DR2, iii) dereddened z-band model
mag\textless{}19.0~AB, and dereddened z-band fiber
mag\textless{}19.5~AB, and 16\textless{}apparent z-band fiber
mag\textless{}19.0~AB, and apparent r-band fiber
mag\textgreater{}16.0~AB, and Gaia G\textgreater{}15.0~Vega, and Gaia
RP\textgreater{}15.0~Vega;


\noindent\textbf{Target priority options:} 7100

\noindent\textbf{Cadence options:} bright\_1x1, dark\_1x1, dark\_1x4

\noindent\textbf{Implementation:}
\href{https://github.com/sdss/target_selection/blob/0.3.13/python/target_selection/cartons/bhm_galaxies.py}{bhm\_galaxies.py}

\noindent\textbf{Number of targets:} 7320203

\begin{center}\rule{0.5\linewidth}{0.5pt}\end{center}

\hypertarget{bhm_spiders_agn_lsdr8_plan0.5.0}{%
\subsection{bhm\_spiders\_agn\_lsdr8}\label{bhm_spiders_agn_lsdr8_plan0.5.0}}

\noindent\textbf{target\_selection plan:} 0.5.0

\noindent\textbf{target\_selection tag:}
\href{https://github.com/sdss/target_selection/tree/0.3.0/}{0.3.0}

\noindent\textbf{Summary:} This is the highest priority carton for SPIDERS AGN
wide area follow up. The carton provides optical counterparts to
point-like (unresolved) X-ray sources detected in early reductions of
the first 6-months of eROSITA all sky survey data (eRASS:1). The sample
is expected to contain a mixture of QSOs, AGN, stars and compact
objects. The X-ray sources have been cross-matched by the eROSITA-DE
team to \href{https://www.legacysurvey.org/dr8/}{legacysurvey.org/dr8}
optical/IR counterparts. All targets are located in the sky hemisphere
where MPE controls the data rights (approx.
180\textless{}l\textless{}360~deg). Due to the footprint of lsdr8,
nearly all targets in this carton are located at high Galactic latitudes
\textbar{}b\textbar{}\textgreater{}15~deg.

\noindent\textbf{Simplified description of selection criteria:} Starting from a
parent catalogue of eRASS:1 point source $\rightarrow$ legacysurvey.org/dr8
associations (method: NWAY assisted by optical/IR priors computed via a
pre-trained Random Forest, building on
\citealt{Salvato2022}), select targets which meet all of the following criteria:
i) have eROSITA detection likelihood\textgreater{}6.0, ii) have an X-ray
$\rightarrow$ optical/IR cross-match probability (NWAY) of p\_any\textgreater{}0.1,
iii) have 13.5\textless{}fibertotmag\_r\textless{}22.5 or
13.5\textless{}fibertotmag\_z\textless{}21.0, iv) are not saturated in
legacysurvey imaging, v) have at least one observation in r-band and at
least one observation in g- or z-band, vi) if detected by Gaia DR2 then
have G\textgreater{}13.5 and RP\textgreater{}13.5~Vega. We deprioritise
targets if any of the following criteria are met: a) the target already
has existing good quality SDSS spectroscopy, b) the X-ray detection
likelihood is \textless{}8.0, c) the target is a secondary
X-ray$\rightarrow$optical/IR association. We assign cadences (exposure time
requests) based on optical brightness.


\noindent\textbf{Target priority options:} 1520-1523, 1720-1723

\noindent\textbf{Cadence options:} bright\_2x1, dark\_1x2, dark\_1x4

\noindent\textbf{Implementation:}
\href{https://github.com/sdss/target_selection/blob/0.3.0/python/target_selection/cartons/bhm_spiders_agn.py}{bhm\_spiders\_agn.py}

\noindent\textbf{Number of targets:} 235745

\begin{center}\rule{0.5\linewidth}{0.5pt}\end{center}

\hypertarget{bhm_spiders_agn_ps1dr2_plan0.5.0}{%
\subsection{bhm\_spiders\_agn\_ps1dr2}\label{bhm_spiders_agn_ps1dr2_plan0.5.0}}

\noindent\textbf{target\_selection plan:} 0.5.0

\noindent\textbf{target\_selection tag:}
\href{https://github.com/sdss/target_selection/tree/0.3.0/}{0.3.0}

\noindent\textbf{Summary:} This is the second highest priority carton for SPIDERS
AGN wide area follow up, it is included to expand the survey footprint
beyond legacysurvey/dr8. The carton provides optical counterparts to
point-like (unresolved) X-ray sources detected in early reductions of
the first 6-months of eROSITA all sky survey data (eRASS:1). The sample
is expected to contain a mixture of QSOs, AGN, stars and compact
objects. The X-ray sources have been cross-matched by the eROSITA-DE
team, first to
CatWISE2020 \citep{Marocco2021}
mid-IR sources, and then to optical counterparts from the
\href{https://outerspace.stsci.edu/display/PANSTARRS/}{PanSTARRS1 dr2}
catalogue. All targets are located in the sky hemisphere where MPE
controls the data rights (approx. 180\textless{}l\textless{}360~deg),
and at Dec\textgreater{}-30~deg, spanning a wide range of Galactic
latitudes. Targets at low Galactic latitudes
\textbar{}b\textbar{}\textless{}15~deg do not drive survey strategy.

\noindent\textbf{Simplified description of selection criteria:} Starting from a
parent catalogue of eRASS:1 point source $\rightarrow$ CatWISE2020 $\rightarrow$ Pan-STARRS1
associations (method: NWAY assisted by IR priors computed via a
pre-trained Random Forest, building on
\citealt{Salvato2022}), select targets which meet all of the following criteria:
i) have eROSITA detection likelihood\textgreater{}6.0, ii) have an
X-ray$\rightarrow$IR cross-match probability of p\_any\textgreater{}0.1, iii) have
psfmag\_g, psfmag\_r, psfmag\_i, psfmag\_z\textgreater{}13.5~AB and at
least one of psfmag\_g\textless{}22.5, psfmag\_r\textless{}22.0,
psfmag\_i\textless{}21.5 or psfmag\_z\textless{}20.5, iv) are not
associated with a bad Pan-STARRS1-dr2 image stack, v) have non-null
measurements of g\_psfmag, r\_psfmag and i\_psfmag, vi) if detected by
Gaia DR2 then have G\textgreater{}13.5 and RP\textgreater{}13.5~Vega. We
deprioritise targets if any of the following criteria are met: a) the
target already has existing good quality SDSS spectroscopy, b) the X-ray
detection likelihood is \textless{}8.0, c) the target is a secondary
X-ray$\rightarrow$IR association. We assign cadences (exposure time requests) based
on optical brightness.


\noindent\textbf{Target priority options:}
1530-1533,1730-1733,3530-3533,3730-3732

\noindent\textbf{Cadence options:} bright\_2x1, dark\_1x2, dark\_1x4

\noindent\textbf{Implementation:}
\href{https://github.com/sdss/target_selection/blob/0.3.0/python/target_selection/cartons/bhm_spiders_agn.py}{bhm\_spiders\_agn.py}

\noindent\textbf{Number of targets:} 200681

\begin{center}\rule{0.5\linewidth}{0.5pt}\end{center}

\hypertarget{bhm_spiders_agn_gaiadr2_plan0.5.0}{%
\subsection{bhm\_spiders\_agn\_gaiadr2}\label{bhm_spiders_agn_gaiadr2_plan0.5.0}}

\noindent\textbf{target\_selection plan:} 0.5.0

\noindent\textbf{target\_selection tag:}
\href{https://github.com/sdss/target_selection/tree/0.3.0/}{0.3.0}

\noindent\textbf{Summary:} This is the third highest priority carton for SPIDERS
AGN wide area follow up, it is included to expand the survey footprint
to the full hemisphere where X-ray sources are available (beyond
legacysurvey/dr8 and Pan-STARRS1). The carton provides optical
counterparts to point-like (unresolved) X-ray sources detected in early
reductions of the first 6-months of eROSITA all sky survey data
(eRASS:1). The sample is expected to contain a mixture of QSOs, AGN,
stars and compact objects. The X-ray sources have been cross-matched by
the eROSITA-DE team, first to
CatWISE2020 \citep{Marocco2021}
mid-IR sources, and then to optical counterparts from the Gaia-dr2
catalogue. All targets are located in the sky hemisphere where MPE
controls the data rights (approx. 180\textless{}l\textless{}360~deg).
The targets in this carton distributed over a wide range of Galactic
latitudes, but targets at low Galactic latitudes
\textbar{}b\textbar{}\textless{}15~deg do not drive survey strategy.

\noindent\textbf{Simplified description of selection criteria:} Starting from a
parent catalogue of eRASS:1 point source $\rightarrow$ CatWISE2020 $\rightarrow$ Gaia DR2
associations (method: NWAY assisted by IR priors computed via a
pre-trained Random Forest, building on
\citealt{Salvato2022}), select targets which meet all of the following criteria:
i) have eROSITA detection likelihood\textgreater{}6.0, ii) have an
X-ray$\rightarrow$IR cross-match probability of p\_any\textgreater{}0.1, iii) have
G\textgreater{}13.5 and RP\textgreater{}13.5~AB. We deprioritise targets
if any of the following criteria are met: a) the target already has
existing good quality SDSS spectroscopy, b) the X-ray detection
likelihood is \textless{}8.0, c) the target is a secondary X-ray$\rightarrow$IR
association. We assign cadences (exposure time requests) based on
optical brightness.


\noindent\textbf{Target priority options:}
1540-1543,1740-1743,3540-3543,3740-3742

\noindent\textbf{Cadence options:} bright\_2x1, dark\_1x2, dark\_1x4

\noindent\textbf{Implementation:}
\href{https://github.com/sdss/target_selection/blob/0.3.0/python/target_selection/cartons/bhm_spiders_agn.py}{bhm\_spiders\_agn.py}

\noindent\textbf{Number of targets:} 324576

\begin{center}\rule{0.5\linewidth}{0.5pt}\end{center}

\hypertarget{bhm_spiders_agn_skymapperdr2_plan0.5.0}{%
\subsection{bhm\_spiders\_agn\_skymapperdr2}\label{bhm_spiders_agn_skymapperdr2_plan0.5.0}}

\noindent\textbf{target\_selection plan:} 0.5.0

\noindent\textbf{target\_selection tag:}
\href{https://github.com/sdss/target_selection/tree/0.3.0/}{0.3.0}

\noindent\textbf{Summary:} This is a lower ranked carton for SPIDERS AGN wide
area follow up, it supplements the survey in areas which rely on
Gaia-DR2 (beyond legacysurvey/dr8 and Pan-STARRS1) by recovering
extended targets (galaxies) that are missed by Gaia. The carton provides
optical counterparts to point-like (unresolved) X-ray sources detected
in early reductions of the first 6-months of eROSITA all sky survey data
(eRASS:1). The sample is expected to contain a mixture of QSOs, AGN,
stars and compact objects. The X-ray sources have been cross-matched by
the eROSITA-DE team, first to
CatWISE2020 \citep{Marocco2021}
mid-IR sources, and then to optical counterparts from the
SkyMapper-dr2 \citep{Onken2019}
catalogue. All targets are located in the sky hemisphere where MPE
controls the data rights (approx. 180\textless{}l\textless{}360~deg) and
at Dec\textless{}0deg, spanning a wide range of Galactic latitudes.
Targets at low Galactic latitudes \textbar{}b\textbar{}\textless{}15~deg
do not drive survey strategy.

\noindent\textbf{Simplified description of selection criteria:} Starting from a
parent catalogue of eRASS:1 point source $\rightarrow$ CatWISE2020 $\rightarrow$ SkyMapper-dr2
associations (method: NWAY assisted by IR priors computed via a
pre-trained Random Forest, building on
\citealt{Salvato2022}), select targets which meet all of the following criteria:
i) have eROSITA detection likelihood\textgreater{}6.0, ii) have an
X-ray$\rightarrow$IR cross-match probability of p\_any\textgreater{}0.1, iii) have
psfmag\_g, psfmag\_r, psfmag\_i, psfmag\_z\textgreater{}13.5~AB and at
least one of psfmag\_g\textless{}22.5, psfmag\_r\textless{}22.0,
psfmag\_i\textless{}21.5 or psfmag\_z\textless{}20.5, iv) Are not
associated with a bad SkyMapper source detection (flags\textless{}4), v)
have non-null measurements in at least one of g\_psfmag, r\_psfmag and
i\_psfmag, vi) if detected by Gaia DR2 then have G\textgreater{}13.5 and
RP\textgreater{}13.5~Vega. We deprioritise targets if any of the
following criteria are met: a) the target already has existing good
quality SDSS spectroscopy, b) the X-ray detection likelihood is
\textless{}8.0, c) the target is a secondary X-ray$\rightarrow$IR association. We
assign cadences (exposure time requests) based on optical brightness.


\noindent\textbf{Target priority options:}
1550-1553,1750-1753,3550-3553,3750-3752

\noindent\textbf{Cadence options:} bright\_2x1, dark\_1x2, dark\_1x4

\noindent\textbf{Implementation:}
\href{https://github.com/sdss/target_selection/blob/0.3.0/python/target_selection/cartons/bhm_spiders_agn.py}{bhm\_spiders\_agn.py}

\noindent\textbf{Number of targets:} 82683

\begin{center}\rule{0.5\linewidth}{0.5pt}\end{center}

\hypertarget{bhm_spiders_agn_supercosmos_plan0.5.0}{%
\subsection{bhm\_spiders\_agn\_supercosmos}\label{bhm_spiders_agn_supercosmos_plan0.5.0}}

\noindent\textbf{target\_selection plan:} 0.5.0

\noindent\textbf{target\_selection tag:}
\href{https://github.com/sdss/target_selection/tree/0.3.0/}{0.3.0}

\noindent\textbf{Summary:} This is a lower ranked carton for SPIDERS AGN wide
area follow up, it supplements the survey in areas which rely on
Gaia-DR2 (beyond legacysurvey/dr8 and Pan-STARRS1) by recovering
extended targets (galaxies) that are missed by Gaia. The carton provides
optical counterparts to point-like (unresolved) X-ray sources detected
in early reductions of the first 6-months of eROSITA all sky survey data
(eRASS:1). The sample is expected to contain a mixture of QSOs, AGN,
stars and compact objects. The X-ray sources have been cross-matched by
the eROSITA-DE team, first to
CatWISE2020 \citep{Marocco2021}
mid-IR sources, and then to optical counterparts from the
\href{http://www-wfau.roe.ac.uk/sss/intro.html}{SuperCosmos Sky Surveys}
catalogue (derived from scans of photographic plates). All targets are
located in the sky hemisphere where MPE controls the data rights
(approx. 180\textless{}l\textless{}360~deg), spanning a wide range of
Galactic latitudes. Targets at low Galactic latitudes
\textbar{}b\textbar{}\textless{}15~deg do not drive survey strategy.

\noindent\textbf{Simplified description of selection criteria:} Starting from a
parent catalogue of eRASS:1 point source $\rightarrow$ CatWISE2020 $\rightarrow$ SuperCosmos
associations (method: NWAY assisted by IR priors computed via a
pre-trained Random Forest, building on
\citealt{Salvato2022}), select targets which meet all of the following criteria:
i) have eROSITA detection likelihood\textgreater{}6.0, ii) have an
X-ray$\rightarrow$IR cross-match probability of p\_any\textgreater{}0.1, iii) have
scormagb, scormagr2, scormagi\textgreater{}13.5~Vega and at least one of
scormagb\textless{}22.5, scormagr2\textless{}22.0, or
scormagi\textless{}21.5, iv) if detected by Gaia DR2 then have
G\textgreater{}13.5 and RP\textgreater{}13.5~Vega. We deprioritise
targets if any of the following criteria are met: a) the target already
has existing good quality SDSS spectroscopy, b) the X-ray detection
likelihood is \textless{}8.0, c) the target is a secondary X-ray$\rightarrow$IR
association. We assign cadences (exposure time requests) based on
optical brightness.


\noindent\textbf{Target priority options:}
1560-1563,1760-1763,3560-3563,3760-3763

\noindent\textbf{Cadence options:} bright\_2x1, dark\_1x2, dark\_1x4

\noindent\textbf{Implementation:}
\href{https://github.com/sdss/target_selection/blob/0.3.0/python/target_selection/cartons/bhm_spiders_agn.py}{bhm\_spiders\_agn.py}

\noindent\textbf{Number of targets:} 430780

\begin{center}\rule{0.5\linewidth}{0.5pt}\end{center}

\hypertarget{bhm_spiders_agn_efeds_stragglers_plan0.5.0}{%
\subsection{bhm\_spiders\_agn\_efeds\_stragglers}\label{bhm_spiders_agn_efeds_stragglers_plan0.5.0}}

\noindent\textbf{target\_selection plan:} 0.5.0

\noindent\textbf{target\_selection tag:}
\href{https://github.com/sdss/target_selection/tree/0.3.0/}{0.3.0}

\noindent\textbf{Summary:} This is an opportunistic supplementary carton for
SPIDERS AGN follow up, aiming, where fiber resources allow, to gather a
small additional number of spectra for targets in the eFEDS field (which
has been repeatedly surveyed in earlier iterations of SDSS). This carton
provides optical counterparts to point-like (unresolved) X-ray sources
detected in early reductions of the eROSITA/eFEDS performance validation
field. The sample is expected to contain a mixture of QSOs, AGN, stars
and compact objects. The X-ray sources have been cross-matched by the
eROSITA-DE team to
\href{https://www.legacysurvey.org/dr8/}{legacysurvey.org/dr8}
optical/IR counterparts. All targets in this carton are located within
the eFEDS field (approx 126\textless{}RA\textless{}146,
-3\textless{}Dec\textless{}+6~deg). These targets do not drive survey
strategy.

\noindent\textbf{Simplified description of selection criteria:} Starting from a
parent catalogue of eFEDS point source $\rightarrow$ legacysurvey.org/dr8
associations (method: NWAY assisted by optical/IR priors computed via a
pre-trained Random Forest, see
\citealt{Salvato2022}), select targets which meet all of the following criteria:
i) have eROSITA detection likelihood\textgreater{}6.0, ii) have an X-ray
$\rightarrow$ optical/IR cross-match probability (NWAY) of p\_any\textgreater{}0.1,
iii) have 13.5\textless{}fibertotmag\_r\textless{}22.5 or
13.5\textless{}fibertotmag\_z\textless{}21.0, iv) are not saturated in
legacysurvey imaging, v) have at least one observation in r-band and at
least one observation in g- or z-band, vi) if detected by Gaia DR2 then
have G\textgreater{}13.5 and RP\textgreater{}13.5~Vega. We deprioritise
targets if any of the following criteria are met: a) the target already
has existing good quality SDSS spectroscopy, b) the X-ray detection
likelihood is \textless{}8.0, c) the target is a secondary
X-ray$\rightarrow$optical/IR association. We assign cadences (exposure time
requests) based on optical brightness.


\noindent\textbf{Target priority options:} 1510-1514, 1710-1714

\noindent\textbf{Cadence options:} bright\_2x1, dark\_1x2, dark\_1x4

\noindent\textbf{Implementation:}
\href{https://github.com/sdss/target_selection/blob/0.3.0/python/target_selection/cartons/bhm_spiders_agn.py}{bhm\_spiders\_agn.py}

\noindent\textbf{Number of targets:} 15926

\begin{center}\rule{0.5\linewidth}{0.5pt}\end{center}

\hypertarget{bhm_spiders_agn_sep_plan0.5.0}{%
\subsection{bhm\_spiders\_agn\_sep}\label{bhm_spiders_agn_sep_plan0.5.0}}

\noindent\textbf{target\_selection plan:} 0.5.0

\noindent\textbf{target\_selection tag:}
\href{https://github.com/sdss/target_selection/tree/0.3.0/}{0.3.0}

\noindent\textbf{Summary:} This special carton is dedicated to SPIDERS AGN follow
up in the CVZ-S/SEP field. The carton provides optical counterparts to
point-like (unresolved) X-ray sources detected in a dedicated analysis
of the first 6-months of eROSITA scanning data near the SEP. The X-ray
sources have been cross-matched by the eROSITA-DE team, first to
CatWISE2020 \citep{Marocco2021}
mid-IR sources, and then to optical counterparts from the Gaia-dr2
catalogue, using additional filtering (including Gaia EDR3 astrometric
info) to reduce the contamination from foreground stars located in the
LMC. All targets are located within 1.5 deg of the SEP (RA,Dec =
90.0,-66.56~deg).

\noindent\textbf{Simplified description of selection criteria:} Starting from a
parent catalogue of eRASS:1/SEP point source $\rightarrow$ CatWISE2020 $\rightarrow$ Gaia DR2
associations (method: NWAY assisted by IR priors computed via a
pre-trained Random Forest, building on
\citealt{Salvato2022}), select targets which meet all of the following criteria:
i) have eROSITA detection likelihood\textgreater{}6.0, ii) have an
X-ray$\rightarrow$IR cross-match probability of p\_any\textgreater{}0.1, iii) have
G\textgreater{}13.5 and RP\textgreater{}13.5~AB. We deprioritise targets
if any of the following criteria are met: a) the X-ray detection
likelihood is \textless{}8.0, b) the target is a secondary X-ray$\rightarrow$IR
association. We assign cadences (exposure time requests) based on
optical brightness.


\noindent\textbf{Target priority options:} 1510,1512

\noindent\textbf{Cadence options:} bright\_2x1, dark\_1x2, dark\_1x4

\noindent\textbf{Implementation:}
\href{https://github.com/sdss/target_selection/blob/0.3.0/python/target_selection/cartons/bhm_spiders_agn.py}{bhm\_spiders\_agn.py}

\noindent\textbf{Number of targets:} 697

\begin{center}\rule{0.5\linewidth}{0.5pt}\end{center}

\hypertarget{bhm_spiders_clusters_lsdr8_plan0.5.0}{%
\subsection{bhm\_spiders\_clusters\_lsdr8}\label{bhm_spiders_clusters_lsdr8_plan0.5.0}}

\noindent\textbf{target\_selection plan:} 0.5.0

\noindent\textbf{target\_selection tag:}
\href{https://github.com/sdss/target_selection/tree/0.3.0/}{0.3.0}

\noindent\textbf{Summary:} This is the highest priority carton for SPIDERS
Clusters wide area follow up. The carton provides a list of galaxies
which are candidate members of clusters selected from early reductions
of the first 6-months of eROSITA all sky survey data (eRASS:1). The
X-ray clusters have been associated by the eROSITA-DE team to
\href{https://www.legacysurvey.org/dr8/}{legacysurvey.org/dr8}
optical/IR counterparts using the eROMAPPER red-sequence finder
algorithm
(\citealt{Rykoff2014};
\citealt{IderChitham2020}). All targets are located in the sky hemisphere
where MPE controls the data rights (approx.
180\textless{}l\textless{}360~deg). Due to the footprint of lsdr8,
nearly all targets in this carton are located at high Galactic latitudes
\textbar{}b\textbar{}\textgreater{}15~deg.

\noindent\textbf{Simplified description of selection criteria:} Starting from a
parent catalogue of eRASS:1 $\rightarrow$ legacysurvey.org/dr8 eROMAPPER cluster
associations, select targets which meet all of the following criteria:
i) have 13.5\textless{}fibertotmag\_r\textless{}21.0 or
13.5\textless{}fibertotmag\_z\textless{}20.0, ii) if detected by Gaia
DR2 then have G\textgreater{}13.5 and RP\textgreater{}13.5~Vega, iii) do
not have existing good quality SDSS spectroscopy. We assign a range of
priorities to targets in this carton, with BCGs top ranked, followed by
candidate member galaxies according their probability of membership. We
assign cadences (exposure time requests) based on optical brightness.


\noindent\textbf{Target priority options:} 1501,1630-1659

\noindent\textbf{Cadence options:} bright\_2x1, dark\_1x2, dark\_1x4

\noindent\textbf{Implementation:}
\href{https://github.com/sdss/target_selection/blob/0.3.0/python/target_selection/cartons/bhm_spiders_clusters.py}{bhm\_spiders\_clusters.py}

\noindent\textbf{Number of targets:} 87490

\begin{center}\rule{0.5\linewidth}{0.5pt}\end{center}

\hypertarget{bhm_spiders_clusters_ps1dr2_plan0.5.0}{%
\subsection{bhm\_spiders\_clusters\_ps1dr2}\label{bhm_spiders_clusters_ps1dr2_plan0.5.0}}

\noindent\textbf{target\_selection plan:} 0.5.0

\noindent\textbf{target\_selection tag:}
\href{https://github.com/sdss/target_selection/tree/0.3.0/}{0.3.0}

\noindent\textbf{Summary:} This is the second highest priority carton for SPIDERS
Clusters wide area follow up, it is designed to expand the survey area
beyond the legacysurvey/dr8 footprint. The carton provides a list of
galaxies which are candidate members of clusters selected from early
reductions of the first 6-months of eROSITA all sky survey data
(eRASS:1). The X-ray clusters have been associated by the eROSITA-DE
team to the
\href{https://outerspace.stsci.edu/display/PANSTARRS/}{PanSTARRS1 dr2}
catalogue using the eROMAPPER red-sequence finder algorithm
(\citealt{Rykoff2014};
\citealt{IderChitham2020}). All targets are located in the sky hemisphere
where MPE controls the data rights (approx.
180\textless{}l\textless{}360~deg). Nearly all targets in this carton
are located at high Galactic latitudes
\textbar{}b\textbar{}\textgreater{}15~deg.

\noindent\textbf{Simplified description of selection criteria:} Starting from a
parent catalogue of eRASS:1 $\rightarrow$ Pan-STARRS1-dr2 eROMAPPER cluster
associations, select targets which meet all of the following criteria:
i) have psfmag\_r, psfmag\_i, psfmag\_z\textgreater{}13.5~AB and at
least one of psfmag\_r\textless{}21.5, psfmag\_i\textless{}21.0 or
psfmag\_z\textless{}20.5, ii) do not have existing good quality SDSS
spectroscopy. We assign a range of priorities to targets in this carton,
with BCGs top ranked, followed by candidate member galaxies according
their probability of membership. We assign cadences (exposure time
requests) based on optical brightness.


\noindent\textbf{Target priority options:} 1502,1660-1689

\noindent\textbf{Cadence options:} bright\_2x1,dark\_1x2,dark\_1x4

\noindent\textbf{Implementation:}
\href{https://github.com/sdss/target_selection/blob/0.3.0/python/target_selection/cartons/bhm_spiders_clusters.py}{bhm\_spiders\_clusters.py}

\noindent\textbf{Number of targets:} 86179

\begin{center}\rule{0.5\linewidth}{0.5pt}\end{center}

\hypertarget{bhm_spiders_clusters_efeds_stragglers_plan0.5.0}{%
\subsection{bhm\_spiders\_clusters\_efeds\_stragglers}\label{bhm_spiders_clusters_efeds_stragglers_plan0.5.0}}

\noindent\textbf{target\_selection plan:} 0.5.0

\noindent\textbf{target\_selection tag:}
\href{https://github.com/sdss/target_selection/tree/0.3.0/}{0.3.0}

\noindent\textbf{Summary:} This is an opportunistic supplementary carton for
SPIDERS Clusters follow up, aiming, where fiber resources allow, to
gather a small additional number of spectra for targets in the eFEDS
field (which has been repeatedly surveyed in earlier iterations of
SDSS). The carton provides a list of galaxies which are candidate
members of clusters selected from early reductions of the eROSITA
performance verification survey in the eFEDS field. The X-ray clusters
have been associated by the eROSITA-DE team to
\href{https://www.legacysurvey.org/dr8/}{legacysurvey.org/dr8}
optical/IR counterparts. All targets in this carton are located within
the eFEDS field (approx 126\textless{}RA\textless{}146,
-3\textless{}Dec\textless{}+6~deg).

\noindent\textbf{Simplified description of selection criteria:} Starting from a
parent catalogue of eFEDS $\rightarrow$
\href{https://www.legacysurvey.org/dr8}{legacysurvey.org/dr8} cluster
associations (eROMAPPER,
\citealt{Rykoff2014};
\citealt{IderChitham2020};
\citealt{Liu2022}), select targets which meet all of the following criteria:
i) have 13.5\textless{}fibertotmag\_r\textless{}21.0 or
13.5\textless{}fibertotmag\_z\textless{}20.0, ii) if detected by Gaia
DR2 then have G\textgreater{}13.5 and RP\textgreater{}13.5~Vega, iii) do
not have existing good quality SDSS spectroscopy. We assign a range of
priorities to targets in this carton, with BCGs top ranked, followed by
candidate member galaxies according their probability of membership. We
assign cadences (exposure time requests) based on optical brightness.


\noindent\textbf{Target priority options:} 1500,1600-1629

\noindent\textbf{Cadence options:} dark\_1x2,dark\_1x4

\noindent\textbf{Implementation:}
\href{https://github.com/sdss/target_selection/blob/0.3.0/python/target_selection/cartons/bhm_spiders_clusters.py}{bhm\_spiders\_clusters.py}

\noindent\textbf{Number of targets:} 3060

\begin{center}\rule{0.5\linewidth}{0.5pt}\end{center}

\hypertarget{bhm_spiders_agn-efeds_plan0.1.0}{%
\subsection{bhm\_spiders\_agn-efeds}\label{bhm_spiders_agn-efeds_plan0.1.0}}

\noindent\textbf{target\_selection plan:} 0.1.0

\noindent\textbf{target\_selection tag:}
\href{https://github.com/sdss/target_selection/tree/0.1.0/}{0.1.0}

\noindent\textbf{Summary:} A carton used during SDSS-V plate-mode observations,
that contains candidate AGN targets found in the eROSITA/eFEDS X-ray
survey field. This carton provides optical counterparts to point-like
(unresolved) X-ray sources detected in early reductions ('c940/V2T') of
the eROSITA/eFEDS performance validation field. The sample is expected
to contain a mixture of QSOs, AGN, stars and compact objects. The X-ray
sources have been cross-matched by the eROSITA-DE team to
\href{https://www.legacysurvey.org/dr8/}{legacysurvey.org/dr8}
optical/IR counterparts. All targets in this carton are located within
the eFEDS field (approx 126\textless{}RA\textless{}146,
-3\textless{}Dec\textless{}+6~deg).

\noindent\textbf{Simplified description of selection criteria:} Starting from a
parent catalogue of eFEDS point source $\rightarrow$
\href{https://www.legacysurvey.org/dr8}{legacysurvey.org/dr8}
associations (primarily via NWAY assisted by optical/IR priors computed
via a pre-trained Random Forest, see
\citealt{Salvato2022}, supplemented by counterparts selected via a Likelihood
Ratio using r-band magnitudes), select targets which meet all of the
following criteria: i) have eROSITA detection
likelihood\textgreater{}6.0, ii) have an X-ray $\rightarrow$ optical/IR cross-match
probability of either p\_any\textgreater{}0.1 (NWAY associations) or
LR\textgreater{}0.2 (Likelihood Ratio associations), iii) have
fibermag\_r\&gt16.5 and fibermag\_r\&lt22.0 or
fibermag\_z\textless{}21.0, and iv) did not receive high quality
spectroscopy during the SDSS-IV= observations of the eFEDS field
(\citealt{Abdurrouf_2021_sdssDR17}). We deprioritise targets if any of the following criteria
are met: a) the target already has existing good quality SDSS
spectroscopy in SDSS DR16, b) the X-ray detection likelihood is
\textless{}8.0, c) the target is a secondary X-ray$\rightarrow$optical/IR
association, or d) if the optical/IR counterpart was only chosen by the
LR method. All targets were assigned a nominal cadence of:
bhm\_spiders\_1x8 (8x15mins dark time).


\noindent\textbf{Target priority options:} 1510-1519

\noindent\textbf{Cadence options:} bhm\_spiders\_1x8

\noindent\textbf{Implementation:}
\href{https://github.com/sdss/target_selection/blob/0.1.0/python/target_selection/cartons/bhm_spiders_agn.py}{bhm\_spiders\_agn.py}

\noindent\textbf{Number of targets:} 12459

\begin{center}\rule{0.5\linewidth}{0.5pt}\end{center}

\hypertarget{bhm_spiders_clusters-efeds-ls-redmapper_plan0.1.0}{%
\subsection{bhm\_spiders\_clusters-efeds-ls-redmapper}\label{bhm_spiders_clusters-efeds-ls-redmapper_plan0.1.0}}

\noindent\textbf{target\_selection plan:} 0.1.0

\noindent\textbf{target\_selection tag:}
\href{https://github.com/sdss/target_selection/tree/0.1.0/}{0.1.0}

\noindent\textbf{Summary:} A carton used during SDSS-V plate-mode observations,
that contains galaxy cluster targets found in the eROSITA/eFEDS X-ray
survey field. The carton provides a list of galaxies which are candidate
members of clusters selected from early reductions ('c940') of the
eROSITA performance verification survey in the eFEDS field. The parent
sample of galaxy clusters and their member galaxies have been selected
via a joint analysis of X-ray and (several) optical/IR datasets using
the eROMAPPER red-sequence finder algorithm
(\citealt{Rykoff2014};
\citealt{IderChitham2020}). This particular carton relies on optical/IR data
from \href{https://www.legacysurvey.org/dr8/}{legacysurvey.org/dr8}. All
targets in this carton are located within the eFEDS field (approx
126\textless{}RA\textless{}146, -3\textless{}Dec\textless{}+6~deg).

\noindent\textbf{Simplified description of selection criteria:} Starting from a
parent catalogue of eFEDS $\rightarrow$ optical/IR cluster associations, select
targets which meet all of the following criteria: i) are selected by
eROMAPPER applied to
\href{https://www.legacysurvey.org/dr8}{legacysurvey.org/dr8}
photometric data, ii) have eROSITA X-ray detection likelihood
\textgreater{} 8.0, iii) have fibermag\_r\textgreater{}16.5 and
fibermag\_r\textless{}21.0 or fibermag\_z\textless{}20.0, iv) do not
have existing good quality (SDSS or external) spectroscopy. We assign a
range of priorities to targets in this carton, with BCGs top ranked,
followed by candidate member galaxies according their probability of
membership. All targets were assigned a nominal cadence of:
bhm\_spiders\_1x8 (8x15mins dark time).


\noindent\textbf{Target priority options:} 1500, 1511-\/-1610

\noindent\textbf{Cadence options:} dark\_1x8

\noindent\textbf{Implementation:}
\href{https://github.com/sdss/target_selection/blob/0.1.0/python/target_selection/cartons/bhm_spiders_clusters.py}{bhm\_spiders\_clusters.py}

\noindent\textbf{Number of targets:} 4432

\begin{center}\rule{0.5\linewidth}{0.5pt}\end{center}

\hypertarget{bhm_spiders_clusters-efeds-sdss-redmapper_plan0.1.0}{%
\subsection{bhm\_spiders\_clusters-efeds-sdss-redmapper}\label{bhm_spiders_clusters-efeds-sdss-redmapper_plan0.1.0}}

\noindent\textbf{target\_selection plan:} 0.1.0

\noindent\textbf{target\_selection tag:}
\href{https://github.com/sdss/target_selection/tree/0.1.0/}{0.1.0}

\noindent\textbf{Summary:} A carton used during SDSS-V plate-mode observations,
that contains galaxy cluster targets found in the eROSITA/eFEDS X-ray
survey field. The carton provides a list of galaxies which are candidate
members of clusters selected from early reductions ('c940') of the
eROSITA performance verification survey in the eFEDS field. The parent
sample of galaxy clusters and their member galaxies have been selected
via a joint analysis of X-ray and (several) optical/IR datasets using
the eROMAPPER red-sequence finder algorithm
(\citealt{Rykoff2014};
\citealt{IderChitham2020}). This particular carton relies on optical/IR data
from \href{https://www.sdss.org/dr13/}{SDSS/dr13}. All targets in this
carton are located within the eFEDS field (approx
126\textless{}RA\textless{}146, -3\textless{}Dec\textless{}+6~deg).

\noindent\textbf{Simplified description of selection criteria:} Starting from a
parent catalogue of eFEDS $\rightarrow$ optical/IR cluster associations, select
targets which meet all of the following criteria: i) are selected by
eROMAPPER applied to \href{https://www.sdss.org/dr13/}{SDSS/dr13}
photometric data, ii) have eROSITA X-ray detection likelihood
\textgreater{} 8.0, iii) have fibermag\_r\textgreater{}16.5 and
fibermag\_r\textless{}21.0 or fibermag\_z\textless{}20.0, iv) do not
have existing good quality (SDSS or external) spectroscopy. We assign a
range of priorities to targets in this carton, with BCGs top ranked,
followed by candidate member galaxies according their probability of
membership. All targets were assigned a nominal cadence of:
bhm\_spiders\_1x8 (8x15mins dark time).


\noindent\textbf{Target priority options:} 1500, 1511-1610

\noindent\textbf{Cadence options:} dark\_1x8

\noindent\textbf{Implementation:}
\href{https://github.com/sdss/target_selection/blob/0.1.0/python/target_selection/cartons/bhm_spiders_clusters.py}{bhm\_spiders\_clusters.py}

\noindent\textbf{Number of targets:} 4304

\begin{center}\rule{0.5\linewidth}{0.5pt}\end{center}

\hypertarget{bhm_spiders_clusters-efeds-hsc-redmapper_plan0.1.0}{%
\subsection{bhm\_spiders\_clusters-efeds-hsc-redmapper}\label{bhm_spiders_clusters-efeds-hsc-redmapper_plan0.1.0}}

\noindent\textbf{target\_selection plan:} 0.1.0

\noindent\textbf{target\_selection tag:}
\href{https://github.com/sdss/target_selection/tree/0.1.0/}{0.1.0}

\noindent\textbf{Summary:} A carton used during SDSS-V plate-mode observations,
that contains galaxy cluster targets found in the eROSITA/eFEDS X-ray
survey field. The carton provides a list of galaxies which are candidate
members of clusters selected from early reductions ('c940') of the
eROSITA performance verification survey in the eFEDS field. The parent
sample of galaxy clusters and their member galaxies have been selected
via a joint analysis of X-ray and (several) optical/IR datasets using
the eROMAPPER red-sequence finder algorithm
(\citealt{Rykoff2014};
\citealt{IderChitham2020}). This particular carton relies on optical/IR data
from the \href{https://hsc.mtk.nao.ac.jp/ssp/}{Hyper Suprime-Cam Subaru
Strategic Program (HSC-SSP)}. All targets in this carton are located
within the eFEDS field (approx 126\textless{}RA\textless{}146,
-3\textless{}Dec\textless{}+6~deg).

\noindent\textbf{Simplified description of selection criteria:} Starting from a
parent catalogue of eFEDS $\rightarrow$ optical/IR cluster associations, select
targets which meet all of the following criteria: i) are selected by
eROMAPPER applied to \href{https://hsc.mtk.nao.ac.jp/ssp/}{HSC-SSP}
photometric data, ii) have eROSITA X-ray detection likelihood
\textgreater{} 8.0, iii) have fibermag\_r\textgreater{}16.5 and
fibermag\_r\textless{}21.0 or fibermag\_z\textless{}20.0, iv) do not
have existing good quality (SDSS or external) spectroscopy. We assign a
range of priorities to targets in this carton, with BCGs top ranked,
followed by candidate member galaxies according their probability of
membership. All targets were assigned a nominal cadence of:
bhm\_spiders\_1x8 (8x15mins dark time).


\noindent\textbf{Target priority options:} 1500, 1511-1610

\noindent\textbf{Cadence options:} dark\_1x8

\noindent\textbf{Implementation:}
\href{https://github.com/sdss/target_selection/blob/0.1.0/python/target_selection/cartons/bhm_spiders_clusters.py}{bhm\_spiders\_clusters.py}

\noindent\textbf{Number of targets:} 924

\begin{center}\rule{0.5\linewidth}{0.5pt}\end{center}

\hypertarget{bhm_spiders_clusters-efeds-erosita_plan0.1.0}{%
\subsection{bhm\_spiders\_clusters-efeds-erosita}\label{bhm_spiders_clusters-efeds-erosita_plan0.1.0}}

\noindent\textbf{target\_selection plan:} 0.1.0

\noindent\textbf{target\_selection tag:}
\href{https://github.com/sdss/target_selection/tree/0.1.0/}{0.1.0}

\noindent\textbf{Summary:} A carton used during SDSS-V plate-mode observations,
that contains galaxy cluster targets found in the eROSITA/eFEDS X-ray
survey field. The carton provides a list of galaxies which are candidate
members of clusters selected from early reductions ('c940') of the
eROSITA performance verification survey in the eFEDS field. The parent
sample of galaxy clusters and their member galaxies have been selected
via a joint analysis of X-ray and (several) optical/IR datasets. This
particular carton includes counterparts to X-ray extended sources that
were not selected by the eROMAPPER red seuence finder algorithm when
applied to any of the legacysurvey/dr8, SDSS/dr13 or HSC-SSP datasets
(i.e. complementary to the cartons:
bhm\_spiders\_clusters-efeds-ls-redmapper,
bhm\_spiders\_clusters-efeds-sdss-redmapper and
bhm\_spiders\_clusters-efeds-hsc-redmapper). All targets in this carton
are located within the eFEDS field (approx
126\textless{}RA\textless{}146, -3\textless{}Dec\textless{}+6~deg).

\noindent\textbf{Simplified description of selection criteria:} Starting from a
parent catalogue of eFEDS $\rightarrow$ optical/IR cluster associations, select
targets which meet all of the following criteria: i) are identified as
being X-ray extended but not selected via the eROMAPPER algorithm, ii)
have eROSITA X-ray detection likelihood \textgreater{} 8.0, iii) have
fibermag\_r\textgreater{}16.5 and fibermag\_r\textless{}21.0 or
fibermag\_z\textless{}20.0, iv) do not have existing good quality (SDSS
or external) spectroscopy. We assign a range of priorities to targets in
this carton, with BCGs top ranked, followed by candidate member galaxies
according their probability of membership. All targets were assigned a
nominal cadence of: bhm\_spiders\_1x8 (8x15mins dark time).


\noindent\textbf{Target priority options:} 1500, 1511-1535

\noindent\textbf{Cadence options:} dark\_1x8

\noindent\textbf{Implementation:}
\href{https://github.com/sdss/target_selection/blob/0.1.0/python/target_selection/cartons/bhm_spiders_clusters.py}{bhm\_spiders\_clusters.py}

\noindent\textbf{Number of targets:} 15
