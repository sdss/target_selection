% 
% \hypertarget{bhm_spiders_agn-efeds_plan0.1.0}{%
% \subsection{bhm\_spiders\_agn-efeds}\label{bhm_spiders_agn-efeds_plan0.1.0}}
% 
% \noindent\textbf{target\_selection plan:} 0.1.0
% 
% \noindent\textbf{target\_selection tag:}
% \href{https://github.com/sdss/target_selection/tree/0.1.0/}{0.1.0}
% 
% \noindent\textbf{Summary:} A carton used during SDSS-V plate-mode observations,
% that contains candidate AGN targets found in the eROSITA/eFEDS X-ray
% survey field. This carton provides optical counterparts to point-like
% (unresolved) X-ray sources detected in early reductions ('c940/V2T') of
% the eROSITA/eFEDS performance validation field. The sample is expected
% to contain a mixture of QSOs, AGN, stars and compact objects. The X-ray
% sources have been cross-matched by the eROSITA-DE team to
% \href{https://www.legacysurvey.org/dr8/}{legacysurvey.org/dr8}
% optical/IR counterparts. All targets in this carton are located within
% the eFEDS field (approx 126\textless RA\textless146,
% -3\textless Dec\textless+6~deg).
% 
% \noindent\textbf{Simplified description of selection criteria:} Starting from a
% parent catalogue of eFEDS point source $\rightarrow$
% \href{https://www.legacysurvey.org/dr8}{legacysurvey.org/dr8}
% associations (primarily via NWAY assisted by optical/IR priors computed
% via a pre-trained Random Forest, see
% \citealt{Salvato2022}, supplemented by counterparts selected via a Likelihood
% Ratio using r-band magnitudes), select targets which meet all of the
% following criteria: i) have eROSITA detection likelihood\textgreater6.0,
% ii) have an X-ray $\rightarrow$ optical/IR cross-match probability of either
% p\_any\textgreater0.1 (NWAY associations) or LR\textgreater0.2
% (Likelihood Ratio associations), iii) have fibermag\_r\&gt16.5 and
% fibermag\_r\&lt22.0 or fibermag\_z\textless21.0, and iv) did not receive
% high quality spectroscopy during the SDSS-IV= observations of the eFEDS
% field
% (\citealt{Abdurrouf_2021_sdssDR17}). We deprioritise targets if any of the following criteria
% are met: a) the target already has existing good quality SDSS
% spectroscopy in SDSS DR16, b) the X-ray detection likelihood is
% \textless8.0, c) the target is a secondary X-ray$\rightarrow$optical/IR association,
% or d) if the optical/IR counterpart was only chosen by the LR method.
% All targets were assigned a nominal cadence of: bhm\_spiders\_1x8
% (8x15mins dark time).
% 
% 
% \noindent\textbf{Target priority options:} 1510-1519
% 
% \noindent\textbf{Cadence options:} bhm\_spiders\_1x8
% 
% \noindent\textbf{Implementation:}
% \href{https://github.com/sdss/target_selection/blob/0.1.0/python/target_selection/cartons/bhm_spiders_agn.py}{bhm\_spiders\_agn.py}
% 
% \noindent\textbf{Number of targets:} 12459
% 
% \begin{center}\rule{0.5\linewidth}{0.5pt}\end{center}
% 
% \hypertarget{bhm_spiders_clusters-efeds-sdss-redmapper_plan0.1.0}{%
% \subsection{bhm\_spiders\_clusters-efeds-sdss-redmapper}\label{bhm_spiders_clusters-efeds-sdss-redmapper_plan0.1.0}}
% 
% \noindent\textbf{target\_selection plan:} 0.1.0
% 
% \noindent\textbf{target\_selection tag:}
% \href{https://github.com/sdss/target_selection/tree/0.1.0/}{0.1.0}
% 
% \noindent\textbf{Summary:} A carton used during SDSS-V plate-mode observations,
% that contains galaxy cluster targets found in the eROSITA/eFEDS X-ray
% survey field. The carton provides a list of galaxies which are candidate
% members of clusters selected from early reductions ('c940') of the
% eROSITA performance verification survey in the eFEDS field. The parent
% sample of galaxy clusters and their member galaxies have been selected
% via a joint analysis of X-ray and (several) optical/IR datasets using
% the eROMAPPER red-sequence finder algorithm
% (\citealt{Rykoff2014};
% \citealt{IderChitham2020}). This particular carton relies on optical/IR data
% from \href{https://www.sdss.org/dr13/}{SDSS/dr13}. All targets in this
% carton are located within the eFEDS field (approx
% 126\textless RA\textless146, -3\textless Dec\textless+6~deg).
% 
% \noindent\textbf{Simplified description of selection criteria:} Starting from a
% parent catalogue of eFEDS $\rightarrow$ optical/IR cluster associations, select
% targets which meet all of the following criteria: i) are selected by
% eROMAPPER applied to \href{https://www.sdss.org/dr13/}{SDSS/dr13}
% photometric data, ii) have eROSITA X-ray detection likelihood
% \textgreater{} 8.0, iii) have fibermag\_r\textgreater16.5 and
% fibermag\_r\textless21.0 or fibermag\_z\textless20.0, iv) do not have
% existing good quality (SDSS or external) spectroscopy. We assign a range
% of priorities to targets in this carton, with BCGs top ranked, followed
% by candidate member galaxies according their probability of membership.
% All targets were assigned a nominal cadence of: bhm\_spiders\_1x8
% (8x15mins dark time).
% 
% 
% \noindent\textbf{Target priority options:} 1500, 1511-1610
% 
% \noindent\textbf{Cadence options:} dark\_1x8
% 
% \noindent\textbf{Implementation:}
% \href{https://github.com/sdss/target_selection/blob/0.1.0/python/target_selection/cartons/bhm_spiders_clusters.py}{bhm\_spiders\_clusters.py}
% 
% \noindent\textbf{Number of targets:} 4304
% 
% \begin{center}\rule{0.5\linewidth}{0.5pt}\end{center}
% 
% \hypertarget{bhm_spiders_clusters-efeds-hsc-redmapper_plan0.1.0}{%
% \subsection{bhm\_spiders\_clusters-efeds-hsc-redmapper}\label{bhm_spiders_clusters-efeds-hsc-redmapper_plan0.1.0}}
% 
% \noindent\textbf{target\_selection plan:} 0.1.0
% 
% \noindent\textbf{target\_selection tag:}
% \href{https://github.com/sdss/target_selection/tree/0.1.0/}{0.1.0}
% 
% \noindent\textbf{Summary:} A carton used during SDSS-V plate-mode observations,
% that contains galaxy cluster targets found in the eROSITA/eFEDS X-ray
% survey field. The carton provides a list of galaxies which are candidate
% members of clusters selected from early reductions ('c940') of the
% eROSITA performance verification survey in the eFEDS field. The parent
% sample of galaxy clusters and their member galaxies have been selected
% via a joint analysis of X-ray and (several) optical/IR datasets using
% the eROMAPPER red-sequence finder algorithm
% (\citealt{Rykoff2014};
% \citealt{IderChitham2020}). This particular carton relies on optical/IR data
% from the \href{https://hsc.mtk.nao.ac.jp/ssp/}{Hyper Suprime-Cam Subaru
% Strategic Program (HSC-SSP)}. All targets in this carton are located
% within the eFEDS field (approx 126\textless RA\textless146,
% -3\textless Dec\textless+6~deg).
% 
% \noindent\textbf{Simplified description of selection criteria:} Starting from a
% parent catalogue of eFEDS $\rightarrow$ optical/IR cluster associations, select
% targets which meet all of the following criteria: i) are selected by
% eROMAPPER applied to \href{https://hsc.mtk.nao.ac.jp/ssp/}{HSC-SSP}
% photometric data, ii) have eROSITA X-ray detection likelihood
% \textgreater{} 8.0, iii) have fibermag\_r\textgreater16.5 and
% fibermag\_r\textless21.0 or fibermag\_z\textless20.0, iv) do not have
% existing good quality (SDSS or external) spectroscopy. We assign a range
% of priorities to targets in this carton, with BCGs top ranked, followed
% by candidate member galaxies according their probability of membership.
% All targets were assigned a nominal cadence of: bhm\_spiders\_1x8
% (8x15mins dark time).
% 
% 
% \noindent\textbf{Target priority options:} 1500, 1511-1610
% 
% \noindent\textbf{Cadence options:} dark\_1x8
% 
% \noindent\textbf{Implementation:}
% \href{https://github.com/sdss/target_selection/blob/0.1.0/python/target_selection/cartons/bhm_spiders_clusters.py}{bhm\_spiders\_clusters.py}
% 
% \noindent\textbf{Number of targets:} 924
% 
% \begin{center}\rule{0.5\linewidth}{0.5pt}\end{center}
% 
% \hypertarget{bhm_spiders_clusters-efeds-erosita_plan0.1.0}{%
% \subsection{bhm\_spiders\_clusters-efeds-erosita}\label{bhm_spiders_clusters-efeds-erosita_plan0.1.0}}
% 
% \noindent\textbf{target\_selection plan:} 0.1.0
% 
% \noindent\textbf{target\_selection tag:}
% \href{https://github.com/sdss/target_selection/tree/0.1.0/}{0.1.0}
% 
% \noindent\textbf{Summary:} A carton used during SDSS-V plate-mode observations,
% that contains galaxy cluster targets found in the eROSITA/eFEDS X-ray
% survey field. The carton provides a list of galaxies which are candidate
% members of clusters selected from early reductions ('c940') of the
% eROSITA performance verification survey in the eFEDS field. The parent
% sample of galaxy clusters and their member galaxies have been selected
% via a joint analysis of X-ray and (several) optical/IR datasets. This
% particular carton includes counterparts to X-ray extended sources that
% were not selected by the eROMAPPER red seuence finder algorithm when
% applied to any of the legacysurvey/dr8, SDSS/dr13 or HSC-SSP datasets
% (i.e. complementary to the cartons:
% bhm\_spiders\_clusters-efeds-ls-redmapper,
% bhm\_spiders\_clusters-efeds-sdss-redmapper and
% bhm\_spiders\_clusters-efeds-hsc-redmapper). All targets in this carton
% are located within the eFEDS field (approx 126\textless RA\textless146,
% -3\textless Dec\textless+6~deg).
% 
% \noindent\textbf{Simplified description of selection criteria:} Starting from a
% parent catalogue of eFEDS $\rightarrow$ optical/IR cluster associations, select
% targets which meet all of the following criteria: i) are identified as
% being X-ray extended but not selected via the eROMAPPER algorithm, ii)
% have eROSITA X-ray detection likelihood \textgreater{} 8.0, iii) have
% fibermag\_r\textgreater16.5 and fibermag\_r\textless21.0 or
% fibermag\_z\textless20.0, iv) do not have existing good quality (SDSS or
% external) spectroscopy. We assign a range of priorities to targets in
% this carton, with BCGs top ranked, followed by candidate member galaxies
% according their probability of membership. All targets were assigned a
% nominal cadence of: bhm\_spiders\_1x8 (8x15mins dark time).
% 
% 
% \noindent\textbf{Target priority options:} 1500, 1511-1535
% 
% \noindent\textbf{Cadence options:} dark\_1x8
% 
% \noindent\textbf{Implementation:}
% \href{https://github.com/sdss/target_selection/blob/0.1.0/python/target_selection/cartons/bhm_spiders_clusters.py}{bhm\_spiders\_clusters.py}
% 
% \noindent\textbf{Number of targets:} 15


\begin{center}\rule{0.5\linewidth}{0.5pt}\end{center}

\hypertarget{bhm_aqmes_med_plan0.1.0}{%
\subsection{bhm\_aqmes\_med}\label{bhm_aqmes_med_plan0.1.0}}

\noindent\textbf{target\_selection plan:} 0.1.0

\noindent\textbf{target\_selection tag:}
\href{https://github.com/sdss/target_selection/tree/0.1.0/}{0.1.0}

\noindent\textbf{Summary:} Spectroscopically confirmed optically bright SDSS
QSOs, selected from the SDSS QSO catalogue (DR16Q,
\citealt{Lyke2020}). Located in 36 mostly disjoint fields within the SDSS QSO
footprint that were pre-selected to contain higher than average numbers
of bright QSOs and CSC targets. The list of field centres can be found
within
\href{https://github.com/sdss/target_selection/blob/0.1.0/python/target_selection/masks/candidate_target_fields_bhm_aqmes_med_v0.2.1.fits}{the
target\_selection repository}.

\noindent\textbf{Simplified description of selection criteria:} Select all
objects from SDSS DR16 QSO catalogue that have
16.0\textless sdss\_psfmag\_i\textless19.1~AB, that lie within
1.49~degrees of at least one AQMES-medium field location.


\noindent\textbf{Target priority options:} 1100

\noindent\textbf{Cadence options:} bhm\_aqmes\_medium\_10x4

\noindent\textbf{Implementation:}
\href{https://github.com/sdss/target_selection/blob/0.1.0/python/target_selection/cartons/bhm_aqmes.py}{bhm\_aqmes.py}

\noindent\textbf{Number of targets:} 2663

\begin{center}\rule{0.5\linewidth}{0.5pt}\end{center}

\hypertarget{bhm_aqmes_med-faint_plan0.1.0}{%
\subsection{bhm\_aqmes\_med-faint}\label{bhm_aqmes_med-faint_plan0.1.0}}

\noindent\textbf{target\_selection plan:} 0.1.0

\noindent\textbf{target\_selection tag:}
\href{https://github.com/sdss/target_selection/tree/0.1.0/}{0.1.0}

\noindent\textbf{Summary:} Spectroscopically confirmed optically faint SDSS QSOs,
selected from the SDSS QSO catalogue (DR16Q,
\citealt{Lyke2020}). Located in 36 mostly disjoint fields within the SDSS QSO
footprint that were pre-selected to contain higher than average numbers
of bright QSOs and CSC targets. The list of field centres can be found
within
\href{https://github.com/sdss/target_selection/blob/0.1.0/python/target_selection/masks/candidate_target_fields_bhm_aqmes_med_v0.2.1.fits}{the
target\_selection repository}.

\noindent\textbf{Simplified description of selection criteria:} Select all
objects from SDSS DR16 QSO catalogue that have
19.1\textless sdss\_psfmag\_i\textless21.0~AB, that lie within
1.49~degrees of at least one AQMES-medium field location.


\noindent\textbf{Target priority options:} 3100

\noindent\textbf{Cadence options:} bhm\_aqmes\_medium\_10x4

\noindent\textbf{Implementation:}
\href{https://github.com/sdss/target_selection/blob/0.1.0/python/target_selection/cartons/bhm_aqmes.py}{bhm\_aqmes.py}

\noindent\textbf{Number of targets:} 16853

\begin{center}\rule{0.5\linewidth}{0.5pt}\end{center}

\hypertarget{bhm_aqmes_wide2_plan0.1.0}{%
\subsection{bhm\_aqmes\_wide2}\label{bhm_aqmes_wide2_plan0.1.0}}

\noindent\textbf{target\_selection plan:} 0.1.0

\noindent\textbf{target\_selection tag:}
\href{https://github.com/sdss/target_selection/tree/0.1.0/}{0.1.0}

\noindent\textbf{Summary:} Spectroscopically confirmed optically bright SDSS
QSOs, selected from the SDSS QSO catalogue (DR16Q,
\citealt{Lyke2020}). Located in 330 fields within the SDSS QSO footprint,
where the choice of survey area prioritized fields that overlapped with
the SPIDERS footprint (approx 180\textless b\textless360~deg), and/or
had higher than average numbers of bright QSOs and CSC targets. The list
of field centres can be found
\href{https://github.com/sdss/target_selection/blob/0.1.0/python/target_selection/masks/candidate_target_fields_bhm_aqmes_wide_v0.2.1.fits}{within
the target\_selection repository}.

\noindent\textbf{Simplified description of selection criteria:} Select all
objects from SDSS DR16 QSO catalogue that have
16.0\textless sdss\_psfmag\_i\textless19.1~AB, and that lie within
1.49~degrees of at least one AQMES-wide2 field location.


\noindent\textbf{Target priority options:} 1210

\noindent\textbf{Cadence options:} bhm\_aqmes\_wide\_2x4

\noindent\textbf{Implementation:}
\href{https://github.com/sdss/target_selection/blob/0.1.0/python/target_selection/cartons/bhm_aqmes.py}{bhm\_aqmes.py}

\noindent\textbf{Number of targets:} 18376

\begin{center}\rule{0.5\linewidth}{0.5pt}\end{center}

\hypertarget{bhm_aqmes_wide2-faint_plan0.1.0}{%
\subsection{bhm\_aqmes\_wide2-faint}\label{bhm_aqmes_wide2-faint_plan0.1.0}}

\noindent\textbf{target\_selection plan:} 0.1.0

\noindent\textbf{target\_selection tag:}
\href{https://github.com/sdss/target_selection/tree/0.1.0/}{0.1.0}

\noindent\textbf{Summary:} Spectroscopically confirmed optically faint SDSS QSOs,
selected from the SDSS QSO catalogue (DR16Q,
\citealt{Lyke2020}). Located in 330 fields within the SDSS QSO footprint,
where the choice of survey area prioritized field that overlapped with
the SPIDERS footprint (approx 180\textless b\textless360~deg), and/or
had higher than average numbers of bright QSOs and CSC targets. The list
of field centres can be found
\href{https://github.com/sdss/target_selection/blob/0.1.0/python/target_selection/masks/candidate_target_fields_bhm_aqmes_wide_v0.2.1.fits}{within
the target\_selection repository}.

\noindent\textbf{Simplified description of selection criteria:} Select all
objects from SDSS DR16 QSO catalogue that have
19.1\textless sdss\_psfmag\_i\textless21.0~AB, and that lie within
1.49~degrees of at least one AQMES-wide2 field location.


\noindent\textbf{Target priority options:} 3210

\noindent\textbf{Cadence options:} bhm\_aqmes\_wide\_2x4

\noindent\textbf{Implementation:}
\href{https://github.com/sdss/target_selection/blob/0.1.0/python/target_selection/cartons/bhm_aqmes.py}{bhm\_aqmes.py}

\noindent\textbf{Number of targets:} 63816

\begin{center}\rule{0.5\linewidth}{0.5pt}\end{center}

\hypertarget{bhm_aqmes_wide3_plan0.1.0}{%
\subsection{bhm\_aqmes\_wide3}\label{bhm_aqmes_wide3_plan0.1.0}}

\noindent\textbf{target\_selection plan:} 0.1.0

\noindent\textbf{target\_selection tag:}
\href{https://github.com/sdss/target_selection/tree/0.1.0/}{0.1.0}

\noindent\textbf{Summary:} Spectroscopically confirmed optically bright SDSS
QSOs, selected from the SDSS QSO catalogue (DR16Q,
\citealt{Lyke2020}). Located in 95 fields within the SDSS QSO footprint,
where the choice of survey area prioritized fields having higher than
average numbers of bright QSOs and CSC targets. The list of field
centres can be found
\href{https://github.com/sdss/target_selection/blob/0.1.0/python/target_selection/masks/candidate_target_fields_bhm_aqmes_wide_v0.2.1.fits}{within
the target\_selection repository}.

\noindent\textbf{Simplified description of selection criteria:} Select all
objects from SDSS DR16 QSO catalogue that have
16.0\textless sdss\_psfmag\_i\textless19.1~AB, and that lie within
1.49~degrees of at least one AQMES-wide3 field location.


\noindent\textbf{Target priority options:} 1200

\noindent\textbf{Cadence options:} bhm\_aqmes\_wide\_3x4

\noindent\textbf{Implementation:}
\href{https://github.com/sdss/target_selection/blob/0.1.0/python/target_selection/cartons/bhm_aqmes.py}{bhm\_aqmes.py}

\noindent\textbf{Number of targets:} 5785

\begin{center}\rule{0.5\linewidth}{0.5pt}\end{center}

\hypertarget{bhm_aqmes_wide3-faint_plan0.1.0}{%
\subsection{bhm\_aqmes\_wide3-faint}\label{bhm_aqmes_wide3-faint_plan0.1.0}}

\noindent\textbf{target\_selection plan:} 0.1.0

\noindent\textbf{target\_selection tag:}
\href{https://github.com/sdss/target_selection/tree/0.1.0/}{0.1.0}

\noindent\textbf{Summary:} Spectroscopically confirmed optically faint SDSS QSOs,
selected from the SDSS QSO catalogue (DR16Q,
\citealt{Lyke2020}). Located in 95 fields within the SDSS QSO footprint,
where the choice of survey area prioritized fields having higher than
average numbers of bright QSOs and CSC targets. The list of field
centres can be found
\href{https://github.com/sdss/target_selection/blob/0.1.0/python/target_selection/masks/candidate_target_fields_bhm_aqmes_wide_v0.2.1.fits}{within
the target\_selection repository}.

\noindent\textbf{Simplified description of selection criteria:} Select all
objects from SDSS DR16 QSO catalogue that have
19.1\textless sdss\_psfmag\_i\textless21.0~AB, and that lie within
1.49~degrees of at least one AQMES-wide2 field location.


\noindent\textbf{Target priority options:} 3200

\noindent\textbf{Cadence options:} dark\_2x4

\noindent\textbf{Implementation:}
\href{https://github.com/sdss/target_selection/blob/0.1.0/python/target_selection/cartons/bhm_aqmes.py}{bhm\_aqmes.py}

\noindent\textbf{Number of targets:} 35803

\begin{center}\rule{0.5\linewidth}{0.5pt}\end{center}

\hypertarget{bhm_aqmes_bonus-dark_plan0.1.0}{%
\subsection{bhm\_aqmes\_bonus-dark}\label{bhm_aqmes_bonus-dark_plan0.1.0}}

\noindent\textbf{target\_selection plan:} 0.1.0

\noindent\textbf{target\_selection tag:}
\href{https://github.com/sdss/target_selection/tree/0.1.0/}{0.1.0}

\noindent\textbf{Summary:} Spectroscopically confirmed SDSS QSOs, with magnitudes
in the range appropriate for observations in dark time, selected from
the SDSS QSO catalogue (DR16Q,
\citealt{Lyke2020}). Located anywhere within the SDSS DR16Q footprint.

\noindent\textbf{Simplified description of selection criteria:} Select all
objects from SDSS DR16 QSO catalogue that have
16.5\textless sdss\_psfmag\_i\textless21.5~AB


\noindent\textbf{Target priority options:} 3300

\noindent\textbf{Cadence options:} dark\_spiders\_1x4

\noindent\textbf{Implementation:}
\href{https://github.com/sdss/target_selection/blob/0.1.0/python/target_selection/cartons/bhm_aqmes.py}{bhm\_aqmes.py}

\noindent\textbf{Number of targets:} 579590

\begin{center}\rule{0.5\linewidth}{0.5pt}\end{center}

\hypertarget{bhm_aqmes_bonus-bright_plan0.1.0}{%
\subsection{bhm\_aqmes\_bonus-bright}\label{bhm_aqmes_bonus-bright_plan0.1.0}}

\noindent\textbf{target\_selection plan:} 0.1.0

\noindent\textbf{target\_selection tag:}
\href{https://github.com/sdss/target_selection/tree/0.1.0/}{0.1.0}

\noindent\textbf{Summary:} Spectroscopically confirmed SDSS QSOs, with magnitudes
in the range appropriate for observations in bright time, selected from
the SDSS QSO catalogue (DR16Q,
\citealt{Lyke2020}). Located anywhere within the SDSS DR16Q footprint.

\noindent\textbf{Simplified description of selection criteria:} Select all
objects from SDSS DR16 QSO catalogue that have
14.0\textless sdss\_psfmag\_i\textless18.0~AB


\noindent\textbf{Target priority options:} 4040

\noindent\textbf{Cadence options:} bhm\_boss\_bright\_3x1

\noindent\textbf{Implementation:}
\href{https://github.com/sdss/target_selection/blob/0.1.0/python/target_selection/cartons/bhm_aqmes.py}{bhm\_aqmes.py}

\noindent\textbf{Number of targets:} 10848

\begin{center}\rule{0.5\linewidth}{0.5pt}\end{center}

\hypertarget{bhm_rm_ancillary_plan0.1.0}{%
\subsection{bhm\_rm\_ancillary}\label{bhm_rm_ancillary_plan0.1.0}}

\noindent\textbf{target\_selection plan:} 0.1.0

\noindent\textbf{target\_selection tag:}
\href{https://github.com/sdss/target_selection/tree/0.1.0/}{0.1.0}

\noindent\textbf{Summary:} A supporting sample of candidate QSOs which have been
selected by the Gaia-unWISE AGN catalog
(\citealt{Shu2019}) and/or the SDSS XDQSO catalog
(\citealt{Bovy2011}). These targets are located within five (+1 backup) well
known survey fields (SDSS-RM, COSMOS, XMM-LSS, ECDFS, CVZ-S/SEP, and
ELIAS-S1).

\noindent\textbf{Simplified description of selection criteria:} Starting from a
parent catalogue of optically selected objects in the RM fields (as
presented by
\citealt{Yang2022}), select candidate QSOs that satisfy all of the
following: i) are identified via external ancillary methods
(photo\_bitmask \& 3 != 0); ii) have
15\textless psfmag\_i\textless21.5~AB; iii) do not have significant
detections (\textgreater3$\sigma$) of parallax and/or proper motion in Gaia
DR2; iv) are not classified as a STAR in SDSS DR16 spectroscopy; and v)
do not lie in the SDSS-RM field


\noindent\textbf{Target priority options:} 1004-1050

\noindent\textbf{Cadence options:} bhm\_rm\_174x8

\noindent\textbf{Implementation:}
\href{https://github.com/sdss/target_selection/blob/0.1.0/python/target_selection/cartons/bhm_rm.py}{bhm\_rm.py}

\noindent\textbf{Number of targets:} 948

\begin{center}\rule{0.5\linewidth}{0.5pt}\end{center}

\hypertarget{bhm_rm_core_plan0.1.0}{%
\subsection{bhm\_rm\_core}\label{bhm_rm_core_plan0.1.0}}

\noindent\textbf{target\_selection plan:} 0.1.0

\noindent\textbf{target\_selection tag:}
\href{https://github.com/sdss/target_selection/tree/0.1.0/}{0.1.0}

\noindent\textbf{Summary:} A sample of candidate QSOs selected via the methods
presented by
\citet{Yang2022}. These targets are located within five (+1 backup) well
known survey fields (SDSS-RM, COSMOS, XMM-LSS, ECDFS, CVZ-S/SEP, and
ELIAS-S1).

\noindent\textbf{Simplified description of selection criteria:} Starting from a
parent catalogue of optically selected objects in the RM fields (as
presented by
\citealt{Yang2022}), select candidate QSOs that satisfy all of the
following: i) are identified via the Skew-T algorithm (skewt\_qso == 1,
and skewt\_qso\_prior == 1 for targets in the CVZ-S/SEP field); ii) have
17\textless psfmag\_i\textless21.5~AB; iii) do not have significant
detections (\textgreater3$\sigma$) of parallax and/or proper motion in Gaia
DR2; iv) are not classified as a STAR in SDSS DR16 spectroscopy; vi)
have detections in all of the gri bands (a Gaia detection is sufficient
in the CVZ-S/SEP field); and vii) do not lie in the SDSS-RM field


\noindent\textbf{Target priority options:} 1002-1050

\noindent\textbf{Cadence options:} bhm\_rm\_174x8

\noindent\textbf{Implementation:}
\href{https://github.com/sdss/target_selection/blob/0.1.0/python/target_selection/cartons/bhm_rm.py}{bhm\_rm.py}

\noindent\textbf{Number of targets:} 3811

\begin{center}\rule{0.5\linewidth}{0.5pt}\end{center}

\hypertarget{bhm_rm_var_plan0.1.0}{%
\subsection{bhm\_rm\_var}\label{bhm_rm_var_plan0.1.0}}

\noindent\textbf{target\_selection plan:} 0.1.0

\noindent\textbf{target\_selection tag:}
\href{https://github.com/sdss/target_selection/tree/0.1.0/}{0.1.0}

\noindent\textbf{Summary:} A sample of candidate QSOs selected via their optical
variability properties, as presented by
\citet{Yang2022}. These targets are located within five (+1 backup) well
known survey fields (SDSS-RM, COSMOS, XMM-LSS, ECDFS, CVZ-S/SEP, and
ELIAS-S1).

\noindent\textbf{Simplified description of selection criteria:} Starting from a
parent catalogue of optically selected objects in the RM fields (as
presented by
\citealt{Yang2022}), select candidate QSOs that satisfy all of the
following: i) have significant variability in the DES or PanSTARRS1
multi-epoch photometry (var\_sn{[}g{]}\textgreater3 and
var\_rms{[}g{]}\textgreater0.05); ii) have
17\textless psfmag\_i\textless20.5~AB; iii) do not have significant
detections (\textgreater3$\sigma$) of parallax and/or proper motion in Gaia
DR2; iv) are not classified as a STAR in SDSS DR16 spectroscopy; and vi)
do not lie in the SDSS-RM field


\noindent\textbf{Target priority options:} 1003-1050

\noindent\textbf{Cadence options:} bhm\_rm\_174x8

\noindent\textbf{Implementation:}
\href{https://github.com/sdss/target_selection/blob/0.1.0/python/target_selection/cartons/bhm_rm.py}{bhm\_rm.py}

\noindent\textbf{Number of targets:} 992

\begin{center}\rule{0.5\linewidth}{0.5pt}\end{center}

\hypertarget{bhm_rm_known_spec_plan0.1.0}{%
\subsection{bhm\_rm\_known\_spec}\label{bhm_rm_known_spec_plan0.1.0}}

\noindent\textbf{target\_selection plan:} 0.1.0

\noindent\textbf{target\_selection tag:}
\href{https://github.com/sdss/target_selection/tree/0.1.0/}{0.1.0}

\noindent\textbf{Summary:} A sample of known QSOs identified through optical
spectroscopy from various projects, as collated by
\citet{Yang2022}. These targets are located within five (+1 backup) well
known survey fields (SDSS-RM, COSMOS, XMM-LSS, ECDFS, CVZ-S/SEP, and
ELIAS-S1).

\noindent\textbf{Simplified description of selection criteria:} Starting from a
parent catalogue of optically selected objects in the RM fields (as
presented by
\citealt{Yang2022}), select targets which satisfy all of the following: i)
are flagged as having a spectroscopic identification in the parent
catalogue; ii) have 15\textless psfmag\_i\textless21.7~AB (SDSS-RM,
CDFS, ELIAS-S1, CVZ-S/SEP fields), or
15\textless psfmag\_i\textless21.5~AB (COSMOS and XMM-LSS fields); iii)
have a spectroscopic redshift in the range 0.005\textless z\textless7;
iv) are not classified as a STAR in SDSS DR16 spectroscopy;


\noindent\textbf{Target priority options:} 1001-1050

\noindent\textbf{Cadence options:} bhm\_rm\_174x8

\noindent\textbf{Implementation:}
\href{https://github.com/sdss/target_selection/blob/0.1.0/python/target_selection/cartons/bhm_rm.py}{bhm\_rm.py}

\noindent\textbf{Number of targets:} 2992

\begin{center}\rule{0.5\linewidth}{0.5pt}\end{center}

\hypertarget{bhm_csc_apogee_plan0.1.0}{%
\subsection{bhm\_csc\_apogee}\label{bhm_csc_apogee_plan0.1.0}}

\noindent\textbf{target\_selection plan:} 0.1.0

\noindent\textbf{target\_selection tag:}
\href{https://github.com/sdss/target_selection/tree/0.1.0/}{0.1.0}

\noindent\textbf{Summary:} X-ray sources from the CSC2.0 source catalogue with
NIR counterparts in 2MASS PSC

\noindent\textbf{Simplified description of selection criteria:} Starting from the
parent catalogue of CSC2.0 sources with optical/IR counterparts
(bhm\_csc). Select entries satisfying the following criteria: i)
counterpart is from the 2MASS catalogue, ii) 2MASS H-band magnitude
measurement is in the accepted range for SDSS-V:
10.0\textless H\textless15.0.


\noindent\textbf{Target priority options:} 4000

\noindent\textbf{Cadence options:} bhm\_csc\_apogee\_3x1

\noindent\textbf{Implementation:}
\href{https://github.com/sdss/target_selection/blob/0.1.0/python/target_selection/cartons/bhm_csc.py}{bhm\_csc.py}

\noindent\textbf{Number of targets:} 10633

\begin{center}\rule{0.5\linewidth}{0.5pt}\end{center}

\hypertarget{bhm_csc_boss_dark_plan0.1.0}{%
\subsection{bhm\_csc\_boss\_dark}\label{bhm_csc_boss_dark_plan0.1.0}}

\noindent\textbf{target\_selection plan:} 0.1.0

\noindent\textbf{target\_selection tag:}
\href{https://github.com/sdss/target_selection/tree/0.1.0/}{0.1.0}

\noindent\textbf{Summary:} X-ray sources from the CSC2.0 source catalogue with
counterparts in Panstarrs1-DR1

\noindent\textbf{Simplified description of selection criteria:} Starting from the
parent catalogue of CSC sources with optical/IR counterparts (bhm\_csc).
Select entries satisfying the following criteria: i) counterpart is from
the PanSTARRS1 catalogue, ii) optical flux/magnitude is in a range
suited to SDSS-V dark time observations:
16.0\textgreater psfmag\_i\textgreater22.0~AB .


\noindent\textbf{Target priority options:} 3000

\noindent\textbf{Cadence options:} bhm\_csc\_boss\_1x4

\noindent\textbf{Implementation:}
\href{https://github.com/sdss/target_selection/blob/0.1.0/python/target_selection/cartons/bhm_csc.py}{bhm\_csc.py}

\noindent\textbf{Number of targets:} 65350

\begin{center}\rule{0.5\linewidth}{0.5pt}\end{center}

\hypertarget{bhm_csc_boss-bright_plan0.1.0}{%
\subsection{bhm\_csc\_boss-bright}\label{bhm_csc_boss-bright_plan0.1.0}}

\noindent\textbf{target\_selection plan:} 0.1.0

\noindent\textbf{target\_selection tag:}
\href{https://github.com/sdss/target_selection/tree/0.1.0/}{0.1.0}

\noindent\textbf{Summary:} X-ray sources from the CSC2.0 source catalogue with
counterparts in Panstarrs1-DR1

\noindent\textbf{Simplified description of selection criteria:} Starting from the
parent catalogue of CSC sources with optical/IR counterparts (bhm\_csc).
Select entries satisfying the following criteria: i) counterpart is from
the PanSTARRS1 catalogue, ii) optical flux/magnitude is in a range
suited to SDSS-V bright time observations:
13.0\textgreater psfmag\_i\textgreater18.0~AB .


\noindent\textbf{Target priority options:} 4000

\noindent\textbf{Cadence options:} bhm\_csc\_boss\_1x1

\noindent\textbf{Implementation:}
\href{https://github.com/sdss/target_selection/blob/0.1.0/python/target_selection/cartons/bhm_csc.py}{bhm\_csc.py}

\noindent\textbf{Number of targets:} 22173

\begin{center}\rule{0.5\linewidth}{0.5pt}\end{center}

\hypertarget{bhm_gua_dark_plan0.1.0}{%
\subsection{bhm\_gua\_dark}\label{bhm_gua_dark_plan0.1.0}}

\noindent\textbf{target\_selection plan:} 0.1.0

\noindent\textbf{target\_selection tag:}
\href{https://github.com/sdss/target_selection/tree/0.1.0/}{0.1.0}

\noindent\textbf{Summary:} A sample of optically faint candidate AGN lacking
spectroscopic confirmations, derived from the parent sample presented by
\citet{Shu2019}, who applied a machine-learning approach to select QSO
candidates from a combination of the Gaia DR2 and unWISE catalogues.

\noindent\textbf{Simplified description of selection criteria:} Starting with the
\citet{Shu2019} catalogue, select targets which satisfy the following
criteria: i) have a Random Forest probability of being a QSO
of\textgreater0.8, ii) are in the (dereddened) magnitude range suitable
for BOSS spectroscopy in dark time
(G\textsubscript{dered}\textgreater16.5 and
RP\textsubscript{dered}\textgreater16.5, as well as
G\textsubscript{dered}\textless21.2 or
RP\textsubscript{dered}\textless21.0,~Vega mag), iii) do not have good
quality optical spectroscopic measurements in SDSS DR16


\noindent\textbf{Target priority options:} 3400

\noindent\textbf{Cadence options:} bhm\_spiders\_1x4

\noindent\textbf{Implementation:}
\href{https://github.com/sdss/target_selection/blob/0.1.0/python/target_selection/cartons/bhm_gua.py}{bhm\_gua.py}

\noindent\textbf{Number of targets:} 2085729

\begin{center}\rule{0.5\linewidth}{0.5pt}\end{center}

\hypertarget{bhm_gua_bright_plan0.1.0}{%
\subsection{bhm\_gua\_bright}\label{bhm_gua_bright_plan0.1.0}}

\noindent\textbf{target\_selection plan:} 0.1.0

\noindent\textbf{target\_selection tag:}
\href{https://github.com/sdss/target_selection/tree/0.1.0/}{0.1.0}

\noindent\textbf{Summary:} A sample of optically bright candidate AGN lacking
spectroscopic confirmations, derived from the parent sample presented by
\citet{Shu2019}, who applied a machine-learning approach to select QSO
candidates from a combination of the Gaia DR2 and unWISE catalogues.

\noindent\textbf{Simplified description of selection criteria:} Starting with the
\citet{Shu2019} catalogue, select targets which satisfy the following
criteria: i) have a Random Forest probability of being a QSO
of\textgreater0.8, ii) are in the (dereddened) magnitude range suitable
for BOSS spectroscopy in bright time
(G\textsubscript{dered}\textgreater13.0 and
RP\textsubscript{dered}\textgreater13.5, as well as
G\textsubscript{dered}\textless18.5 or
RP\textsubscript{dered}\textless18.5,~Vega mags), iii) do not have good
quality optical spectroscopic measurements in SDSS DR16.


\noindent\textbf{Target priority options:} 4040

\noindent\textbf{Cadence options:} bhm\_boss\_bright\_3x1

\noindent\textbf{Implementation:}
\href{https://github.com/sdss/target_selection/blob/0.1.0/python/target_selection/cartons/bhm_gua.py}{bhm\_gua.py}

\noindent\textbf{Number of targets:} 237236
